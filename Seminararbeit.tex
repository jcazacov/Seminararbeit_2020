\documentclass[a4paper,12pt]{article} % Dokunmentklasse
%%% Arbeit mit Sprache
\usepackage{cmap}					% Suche im PDF
\usepackage[T2A]{fontenc}			% Kodierung
\usepackage[utf8]{inputenc}			% Kodierung Ursprungstext
\usepackage[ngerman]{babel}	% Sprache

\usepackage{hyperref} %Internet URLs

%%% Erweiterungen für Mathematik
\usepackage{amsmath,amsfonts,amssymb,amsthm,mathtools} % AMS
\usepackage{icomma} % Intelligent comma

%% Formelnummer
\mathtoolsset{showonlyrefs=true} % Zeigt die Nummer der Formel nur wenn mit \eqref{} auf sie im Text verwiesen wird.

%%% Seite
\usepackage{extsizes} % Mehr verschiedener Schriftgrößen
\usepackage{geometry} % Einstellung der Ränder
\geometry{top=25mm}
\geometry{bottom=25mm}
\geometry{left=35mm}
\geometry{right=25mm}

\usepackage{fancyhdr} % Kopf- und Fußzeile
\pagestyle{fancy}
\renewcommand{\headrulewidth}{0mm}  % Dicke der Linie unter der Kopfzeile
%\lfoot{Unten Links}
%\rfoot{Unten Rechts}
%\rhead{Oben Rechts}
%\chead{Oben Mitte}
%\lhead{Oben Links}
% \cfoot{Unten Mitte} % falls nicht angegeben: Seitennummer

\usepackage{setspace} % Zeilenabstand
\onehalfspacing % Zeilenabstand 1.5
%\doublespacing % Zeilenabstand 2
%\singlespacing % Zeilenabstand 1

%% Schriften
\usepackage{euscript}
%\usepackage{times}
%\usepackage{mathptmx} %times
\usepackage{mathrsfs} % Schöne Matheschrift

%% Zeilenumbruch in Formeln
\newcommand*{\hm}[1]{#1\nobreak\discretionary{}
{\hbox{$\mathsurround=0pt #1$}}{}}

%%% Kopfzeile
\author{Jacob Cazacov}
\title{Komplexe Zahlen von algebraischer Sicht aus betrachtet}
\date{\today}

\usepackage{cite} % Bibliographie
%\usepackage[superscript]{cite} % Verweise in obere Zeilen
%\usepackage[nocompress]{cite} % 
\usepackage{csquotes} % Weiteres Werkzeug für Verweise


\usepackage{pgfplots} %Geogebra
\pgfplotsset{compat=1.15}
\usepackage{mathrsfs}
\usetikzlibrary{arrows}
\usepackage{siunitx}





\begin{document} % Dokumentbeginn

\maketitle

\thispagestyle{empty}

\newpage

\thispagestyle{empty}

%\renewcommand{\contentsname}{Gliederung}	% Überschrift des Inhaltsverzeichnisses

\tableofcontents

\newpage


\section{Einleitung}

Die Einführung der komplexen Zahlen war eine der wichtigsten Errungenschaft der Mathematik.
Die Idee, dass man einen neuen Zahlenbereich benutzen könnte, welcher auf den ersten Blick keinerlei Realitätsbezug habe, wurde lange Zeit von Mathematikern aller Welt als sinnlos gesehen.\footnote{Vgl. Pieper: Die komplexen Zahlen. S. $193$.}
Erst als das entdeckerische Verlangen, alle Geheimnisse unserer Welt über Zahlen zu lüften, die Wissenschaftler an das Ende aller anschaulicher und einfacher Mathematik trieb, waren sie bereit, diesen Sprung zu wagen.\footnote{Vgl. Courant/Robbins: Was ist Mathematik? S. $71$.}
Und mit erstaunlichem Erfolg.
Denn Komplexe Zahlen bieten nicht nur einen Lösungsweg für Polynome höheren Grades, sondern können große Teile der Geometrie sehr einfach in rein algebraischer Form ausdrücken, mit welcher man viel bequemer umgehen kann.
Insgesamt bildet das Rechnen in $\mathbb{C}$, dem Zahlenbereich der komplexen Zahlen, ein sehr wichtiges Werkzeug für eine Vielzahl von Naturwissenschaften.

\section{Geschichte und Definition}

Der Zahlenbereich der reellen Zahlen war lange Zeit lang der größte und am meisten umfassendste Zahlenbereich, den die Menschheit kannte.
Er leistete auch sehr gute Dienste, denn mit ihm konnten nahezu alle algebraische Probleme gelöst werden und für Jahrtausende war dies genügend.
Umso provozierender war die Gleichung 
\[x^2+1=0\]
welche keine Lösung in den reellen Zahlen hatte, da es in ihnen keine Wurzel aus $-1$ gibt.\footnote{Vgl. Engel/Fest: Komplexe Zahlen und ebene Geometrie. S. $6$.}
Das Problem ließ sich leicht umgehen, indem einfach festgelegt wurde, dass diese Gleichung keine Lösung habe, es war jedoch nicht aus der Welt geschafft.
Im Jahre 1545 veröffentlichte Geronimo Cardano (1501-1576) in seinem Werk \emph{"{Ars} magna de Regulis Algebraicis"} ein Verfahren, mit dem sich kubische Gleichungen der Form $ax^3+bx^2+cx+d=0$ lösen ließen.
Dieses Verfahren hatte jedoch ein Problem, als Zwischenergebnisse mussten teilweise Wurzeln aus negativen Zahlen gezogen werden.\footnote{Vgl. ebd. S. $76 f.$.} 
%\footnote{Vgl. Engel, Joachim; Fest, Andreas: Komplexe Zahlen und ebene Geometrie, Ludwigsburg $2009^3$, S. $76 f.$}

Im Laufe der Zeit stieß man immer öfter auf solche Schwierigkeiten in einer Vielzahl von verschiedenen Aufgaben und dies führte dazu, dass einige Mathematiker sich entschlossen das Problem direkt anzugehen.

Somit wurde die imaginäre Einheit $i$ als
\[i^2=-1\]
definiert, wobei für $i$ dieselben arithmetischen Regeln gelten, wie für alle anderen Zahlen.\footnote{Vgl. Schmidt: Elementare Technomathematik. S. $285$.}


Dieses $i$ eröffnet nun zahlreiche Möglichkeiten.
Zuallererst natürlich entstanden nun Zahlen der Form $z=a+ib$, die sogenannten komplexen Zahlen.
Sie konnten nun in allen möglichen Bereichen der Mathematik und sogar Physik angewandt werden.\footnote{Vgl. Stewart: Welt-formeln. S. $143$.}
$\mathbb{C}$ beschreibt ihre Gesamtheit und wurde somit zum neuen besten Zahlenbereich zum Lösen von Polynomen.


\section{Algebraische Abgeschlossenheit von $\mathbb{C}$}

Wie schon vorher erwähnt wurde, dienten die komplexen Zahlen ursprünglich als Mittel zur Lösung von Polynomen hohen Grades.
Da dies ihr wichtigster Verwendungszweck war, beschreibt auch einer der wichtigsten Sätze der Mathematik eben diese Eigenschaft.
Der sogenannte Fundamentalsatz der Algebra wurde 1799 von Carl Friedrich Gauß in seiner Dissertation bewiesen und lautet:\\

\noindent "\uppercase{Satz von Gauss:} \emph{Jede algebraische Gleichung $n$-ten Grades mit reellen (oder komplexen Koeffizienten)}
\begin{equation}\label{eq:funda}
		f(x)=x^n + a_{n-1}x^{n-1} + a_{n-2}x^{n-2} + \dots + a_1x + a_0 = 0
\end{equation} 
\noindent \emph{hat im Körper der komplexen Zahlen mindestens eine Lösung, also eine komplexe Zahl $\alpha$ mit f($\alpha$)=0.}"{} 
\footnote{Pieper: Die komplexen Zahlen. S. $178$.}\\

Mit diesem Satz lassen sich jegliche Polynome in den komplexen Zahlen lösen und er hat somit eine grundlegende Bedeutung für die Mathematik.
Der Fundamentalsatz der Algebra wird manchmal auch in einer leicht verschiedenen, mehr umfassenden Form ausgedrückt, welche sich aus der ersten Formulierung einfach herausleiten lässt.\\

Für den Beweis dieser zweiten Fassung des Fundamentalsatzes muss zuerst ein beliebiges Polynom $f(x)$ der $n$-ten Ordnung durch den Term $(x-a)$ geteilt werden.
\begin{equation}\label{eq:bezout1}
f(x) = (x - a) \cdot q(x) + r(x)
\end{equation} 
Dabei ist $a$ eine beliebige Zahl, $q(x)$ das entstehende Polynom der Ordnung $n-1$ und $r(x)$ der Rest, der aus dieser Teilung hervorgeht.
Da $f(x)$ durch ein Polynom ersten Grades geteilt wird, muss der Grad dieses Restes $r(x)$ kleiner als $1$ sein und $r(x)$ ist somit eine Konstante.
Setzt man nun in die obenstehende Gleichung \eqref{eq:bezout1} ein $x=a$, dann folgt:
\[ f(a) = (a-a) \cdot q(a) + r(a) \]
Und da $(a - a) \cdot q(a) = 0 \cdot q(a) = 0 $ ist gilt:
\[ f(a) = r(a) \]
In Worten ausgedrückt:

\noindent \emph{Teilt man ein Polynom $f(x)$ durch den Term $(x-a)$, so ist der Rest, der in dieser Division entsteht, gleich dem Wert der Funktion $f(x)$ bei $x=a$.}\footnote{Vgl. Ansorge/Oberle/Rothe/Sonar: Mathematik für Ingenieure. S.$41$.}\\

Diese Tatsache hat im Deutschen keinen eigenen Namen, ist jedoch im englischsprachigen Raum als das "Little Bézout's theorem" {} bekannt.\footnote{Wikipedia: Polynomial remainder theorem.}

Sei nun $x_0$ eine Nullstelle des Polynoms $f(x)$, also $f(x_0)=0$. Dann folgt aus dem "Little Bézout's theorem" {}in diesem Spezialfall:


\begin{equation}\label{eq:bezout0}
r(x_0) = f(x_0)=0
\end{equation}


\noindent \emph{Ist $x_0$ eine Nullstelle des Polynom $f(x)$, so lässt sich $f(x)$ durch das lineare Polynom $(x-x_0)$ restlos teilen.\footnote{Vgl. Ansorge/Oberle/Rothe/Sonar: Mathematik für Ingenieure. S.$41$.}}\\


%\noindent \emph{Demnach lässt sich ein beliebiges Polynom f(x) durch ein lineares Polynom $(x-x_0)$ mit $x_0$ als eine Nullstelle restlos teilen.}\\

Wenn nun ein Polynom $f(x)$ des $n$-ten Grades vorliegt, so besagt die erste Fassung des Fundamentalsatzes der Algebra \eqref{eq:funda}, dass eine Nullstelle $x_0$ existiert. Folglich lässt sich nun unser $f(x)$ mithilfe des "Little Bézout's theorem" {} umformen zu
\[f(x)=(x-x_0) \cdot g(x)\]
wobei sich f(x) restlos durch $(x-x_0)$ teilt und $g(x)$ ein Polynom der Ordnung $n-1$ ist.
Dieses $g(x)$ hat jedoch nach dem Fundamentalsatz auch eine Nullstelle $x_1$.
Es lässt sich durch $(x-x_1)$ restlos teilen und damit folgt $f(x) = (x-x_0) \cdot (x-x_1) \cdot h(x)$.
Dieselbe Überlegungen können bei $h(x)$ angewandt werden und so weiter, wobei der Grad der Funktion in Term bei jedem Schritt sich um $1$ verkleinert.
Insgesamt lässt sich diese Methode auf dem Polynom $n$ mal verwenden und es teilt sich ohne Rest in $n$ Faktoren auf.\footnote{Vgl. Schmidt: Elementare Technomathematik. S. $322$.}\\

Hiermit kann jedes Polynom $n$-ten Grades in $n$ Faktoren zerteilt werden.
Dabei bildet es die Form
\begin{equation}\label{eq.faktor}
	\boxed{f(x)=(x-x_0)(x-x_1)(x-x_2)\dots(x-x_{n-1})}
\end{equation}
mit den komplexen Zahlen $x_0, x_1, x_2,\dots,x_{n-1}$ als Nullstellen des Polynoms.\\




Als Folgerung hieraus kann nun eine zweite Fassung des Fundamentalsatzes der Algebra gebildet werden:\\

\noindent Jede Gleichung folgender Form (wobei $n \geq 1$):
\begin{equation}\label{eq:funda2}
		a_nz^n + a_{n-1}z^{n-1} + a_{n-2}z^{n-2} + \dots + a_1z + a_0 = 0	
\end{equation}
besitzt \underline{genau $n$} Lösungen in $\mathbb{C}$, wobei die Vielfachheit der Lösung beachtet werden muss.\footnote{Vgl. Courant/Robbins: Was ist Mathematik? S. $80$.}\\





Aus dieser Tatsache folgt direkt die algebraische Abgeschlossenheit von $\mathbb{C}$, nämlich dass jedes Polynom sich in diesem Zahlenbereich lösen lässt.
\footnote{Vgl. Schmidt: Elementare Technomathematik. S. $307$.} 
Komplexe Zahlen wurden somit das Mittel schlechthin im Umgang mit Polynomen, sie endeten viele Probleme und eröffneten viele Möglichkeiten.
Diese Zahlen fanden Anwendung bei der Mathematik der Reste und hatte wegweisende Bedeutung für die Zahlentheorie.
Sie sind zweifellos die Grundlage für viele Teilbereiche der Mathematik. 
\footnote{Vgl. Pieper: Die komplexen Zahlen. S. $199$.}


\section{Eulersche Formel}

Die komplexen Zahlen sind jedoch nicht nur zur Lösung von Polynomen verwendbar, sondern werden auch bei Aufgaben benutzt, die auf den ersten Blick gar nicht mit traditioneller Algebra verbunden sind.
Um diese Aspekte zu erkennen, müssen wir jedoch zuerst ihre geometrische Darstellung betrachten.\\

Jede komplexe Zahl $z$ lässt sich wie wir wissen in der Form $z=a+bi$ schreiben, mit $a$ und $b$ als reelle Zahlen.
Hierbei besteht sie aus zwei Teilen.
Das $a$ wird als der "Realteil" {} $Re(z)$ der Zahl $z$ bezeichnet, dagegen ist $b$ der "{Imaginärteil}" {} $Im(z)$.\footnote{Vgl. Ansorge/Oberle/Rothe/Sonar: Mathematik für Ingenieure. S.33.}
Geometrisch werden solche Zahlen auf leicht unerwartete Weise dargestellt, welche erst einmal erfunden werden musste.\\

Die reellen Zahlen sind offensichtliche ein Spezialfall der komplexen Zahlen, nämlich wenn der Imaginärteil gleich null ist.
Damit kann der Wert einer reellen Zahl als der Realteil einer komplexen Zahl $z$ mit $Im(z)=0$ angesehen werden.
Da alle reelle Zahlen als Punkte auf einer Zahlengerade visualisiert werden und $\mathbb{R}$ eine Teilmenge von $\mathbb{C}$ ist, muss auch in der geometrischen Form der komplexen Zahlen die reelle Zahlengerade vorhanden sein.
Der Imaginärteil wird dagegen in der vollkommen verschiedenen imaginären Einheit $i$ angegeben und lässt sich somit nicht auf dieser Zahlengerade darstellen.
Um diese beiden Bedingungen zu erfüllen wurde eine elegante Lösung gefunden:\\


\noindent \emph{Bei der komplexen Zahlenebene, auch Gaußsche Zahlenebene genannt, handelt es sich um ein kartesisches Koordinatensystem mit einer Achse, welche die Größe des Realteils $Re(z)$ der komplexen Zahl angibt, und der zweiten Achse für den Imaginärteil $Im(z)$. Jede komplexe Zahl wird als Punkt in dieser Ebene veranschaulicht.\footnote{Vgl. Schmidt: Elementare Technomathematik. S. $286$.}}\\


Somit ist die Realachse gleichzeitig auch die Zahlengerade von $\mathbb{R}$, da für jeden Punkt auf ihr gilt $Im(z)=0$.\\



Wir können nun komplexe Zahlen als Punkte in einem Koordinatensystem betrachten.
Statt der klassischen Angabe der Position mit zwei Werten, einen für jede Achse, kann jeder Punkt auch durch seine Polarkoordinaten definiert werden.
Dabei wird von einem Vektor, der im Ursprung beginnt und an dem zu beschreibenden Punkt endet, dessen Winkel $\varphi$ zur Realachse und die Länge dieses Vektors $r$ angegeben. Diese Länge $r$ wird der Betrag $r=|z|$ der komplexen Zahl $z$ genannt.
Der "Winkel [$\varphi$] heißt das Argument oder die Phase, der komplexen Zahl $z$."\footnote{Ansorge/Oberle/Rothe/Sonar: Mathematik für Ingenieure. S.$35$.}

In einem solchen Fall wäre bei der Realteil der Zahl gleich $Re(z)=r\cdot \cos(\varphi)$ und der Imaginärteil $Im(z)=r\cdot \sin(\varphi)$.
Jede komplexe Zahl ist also gleich

\begin{equation}\label{eq.trigon}
	\boxed{z=r\cdot(\cos\varphi+i \sin\varphi)}	
\end{equation}

\begin{center}
	\definecolor{qqwuqq}{rgb}{0,0.39215686274509803,0}
	\definecolor{ududff}{rgb}{0.30196078431372547,0.30196078431372547,1}
	\definecolor{uuuuuu}{rgb}{0.26666666666666666,0.26666666666666666,0.26666666666666666}
	\begin{tikzpicture}[line cap=round,line join=round,>=triangle 45,x=1cm,y=1cm]
		\begin{axis}[
			x=1cm,y=1cm,
			axis lines=middle,
			ymajorgrids=true,
			xmajorgrids=true,
			xlabel={$Re(z)$},
			ylabel={$Im(z)$},
			scale=1.5,
			xmin=-1.5,
			xmax=4.5,
			ymin=-1.5,
			ymax=3.5,
			xtick={-1,0,...,4},
			ytick={-1,0,...,3},]
			\clip(-5.333670383259202,-8.150552521109908) rectangle (11.738093543694342,6.020399486548278);
			\draw [shift={(0,0)},line width=2pt,color=qqwuqq,fill=qqwuqq,fill opacity=0.10000000149011612] (-1.5,-1.5) -- (-100:1.52) arc (0:42:0.9715638007209334) -- cycle;
			\draw [->,line width=2.4pt] (0,0)-- (3,2);
			\draw [line width=1.6pt,dash pattern=on 1pt off 4pt] (0,2)-- (3,2);
			\draw [line width=1.6pt,dash pattern=on 1pt off 4pt] (3,2)-- (3,0);
			\begin{scriptsize}
				\draw [fill=uuuuuu] (0,0) circle (2pt);
				\draw [fill=ududff] (3,2) circle (2.5pt);
				\draw[color=ududff] (3.6,2.3) node {\large $z = (3, 2)$};
				\draw[color=qqwuqq] (1.5,0.49636530530639955) node {\large $\varphi$};
				\draw[color=black] (1.3,1.15) node {\large $r$};
				\draw[color=black] (1.5,2.3) node {\large $r\cdot \sin(\varphi)$};
				\draw[color=black] (3.8,0.95) node {\large $r \cdot \cos(\varphi)$};
			\end{scriptsize}
		\end{axis}
	\end{tikzpicture}

\begin{small}
	Abbildung $1$: Gaußsche Zahlenebene
\end{small}

\end{center}

Betrachten wir nun den Sinus, beziehungsweise den Kosinus.
Wollen wir sie mit anderen Funktionen vergleichen, so wäre es besser wenn sie in der Polynomform vorlägen, da mit dieser in der Algebra leicht umgegangen werden kann.
Um dies zu erreichen benötigen wir das Taylorpolynom.
Dieses Polynom wird oft in der Physik verwendet, wie zum Beispiel in der Formel zu Berechnung der zurückgelegten Strecke bei gleichmäßig beschleunigter Bewegung. 

\[s(t)=s_0+v\cdot t + \frac{1}{2}at^2=\frac{s(t_0)}{0!}+\frac{s\prime(t_0)}{1!}\cdot t +\frac{s\prime\prime(t_0)}{2!} \cdot t^2\]


Das Taylorpolynom ist perfekt dazu geeignet komplizierte Funktionen mit hoher Genauigkeit anzunähern.
Das Polynom lässt sich nur dann anwenden, wenn die anzunähernde Funktion sich zureichend oft ableiten lässt, am besten unbeschränkt oft.
Für sowohl den Sinus als auch den Kosinus ist letzteres wahr.
Bei wiederholter Ableitung laufen sie durch die Werte $\sin(x) \rightarrow \cos(x) \rightarrow -\sin(x) \rightarrow -\cos(x) \rightarrow \sin(x)$ unendlich oft durch.
Das Taylorpolynom kann mit einer angegebenen Entwicklungsstelle $x_0$ den Wert jeder Funktion $f(x)$ an der Position $x$ durch die Formel
\begin{equation}
	T_n(x)=\frac{f(x_0)}{0!}\cdot(x-x_0)^0 + \frac{f^{(1)}(x_0)}{1!}\cdot(x-x_0)^1 +% \frac{f^{(2)}(x_0)}{2!}\cdot(x-x_0)^2 +
	\dots +\frac{f^{(n)}(x_0)}{n!}\cdot(x-x_0)^n
\end{equation} %Video von Mathepeter
annähern, wobei bei steigender Anzahl $n$ der Polynomglieder die Genauigkeit der Annäherung steigt.\footnote{Vgl. Bronstein/Mühlig/Musiol/Semendjajew: Taschenbuch der Mathematik. S. $455$.}


Versuchen wir nun den $\cos(x)$ durch ein unendlich langes Taylorpolynom, eine sogenannte Taylorreihe, so sehr anzunähern, dass das Polynom getrost der Funktion gleichgesetzt werden kann.


Wir bestimmen $x_0=0$, da an dieser Stelle die Sinus- und Kosinuswerte einfach bestimmt werden können.
\[\cos(x)=\frac{\cos(0)}{0!}\cdot(x-0) + \frac{-\sin(0)}{1!}\cdot(x-0) + \frac{-\cos(0)}{2!}\cdot(x-0)^2+\frac{\sin(0)}{3!}\cdot(x-0)^3+\dots\]
Dabei fällt auf, dass bei jedem zweiten Polynomglied im Zähler des Bruches ein positives oder negatives $\sin(0)$ vorliegt.
Da $\sin(0)=0$ ist, haben diese Glieder keinen Einfluss auf den Wert des Polynoms und können weggelassen werden. In den restlichen Gliedern ligt ein $\cos(x)$ vor. Dies ist bei $x=0$ gleich $\cos(0)=1$. Damit lässt sich das Polynom vereinfachen zu

\begin{equation}\label{cos}
	\cos(x)=1-\frac{x^2}{2!}+\frac{x^4}{4!}-\frac{x^6}{6!}+\dots
\end{equation}

Nach analogem Verfahren erhält man auch die Sinusfunktion als Reihe

\begin{equation}\label{sin}
	\sin(x)=\frac{x}{1!}-\frac{x^3}{3!}+\frac{x^5}{5!}-\frac{x^7}{7!}+\dots
\end{equation}

Mithilfe des Sinus und Kosinus können wir nun komplexe Zahlen als Polynome aufschreiben. Sehen wir uns der Einfachheit halber zuerst eine Zahl an, die auf dem Einheitskreis liegt und damit als $r=1$ hat. Setzen wir \eqref{cos} und \eqref{sin} in \eqref{eq.trigon}, so folgt
\begin{equation}\label{taylor}
	z=%\cdot(\cos(\varphi)+i\cdot \sin(\varphi))=
	\left(1-\frac{x^2}{2!}+\frac{x^4}{4!}-\frac{x^6}{6!}+\dots\right)+i\left(\frac{x}{1!}-\frac{x^3}{3!}+\frac{x^5}{5!}-\frac{x^7}{7!}+\dots\right)
\end{equation}

Entwickeln wir nun auch die Funktion $e(x)$ als Taylorreihe, im Punkt $x_0=0$.
Dabei beachten wir, dass $e(0)=1$ und $e\prime(x)=e(x)$.

\begin{equation}\label{expo}
	e^x=1+\frac{x}{1!}+\frac{x^2}{2!}+\frac{x^3}{3!}+\frac{x^4}{4!}+\dots
\end{equation}




Setzen wir in \eqref{taylor} als $x=\varphi$ und in \eqref{expo} $x=i\varphi$ ein.
Bedenken wir, dass $i^2=-1$, $i^3=-i$ und $i^4=1$, entsteht durch Umformungen und Umgruppierungen:\footnote{Mathehilfe: Exponentialform der komplexen Zahlen.}

\begin{multline}
	e^{i\varphi}=1+\frac{i\varphi}{1!}+\frac{i^2\varphi^2}{2!}+\frac{i^3\varphi^3}{3!}+\frac{i^4\varphi^4}{4!}+\dots=
	1+\frac{i\varphi}{1!}-\frac{\varphi^2}{2!}-\frac{i\varphi^3}{3!}+\frac{\varphi^4}{4!}+\dots=\\
	\left(1-\frac{\varphi^2}{2!}+\frac{\varphi^4}{4!}-\frac{\varphi^6}{6!}+\dots\right)+i\cdot\left(\frac{\varphi}{1!}-\frac{x\varphi^3}{3!}+\frac{\varphi^5}{5!}-\frac{\varphi^7}{7!}+\dots\right) =\\
	\cos(\varphi) + i\cdot \sin(\varphi)=z
\end{multline}
Hierbei handelt es sich um die berühmte \emph{"{}Eulersche Formel"}, welche besagt, dass
\begin{equation} \label{euler}
	e^{i\varphi}=\cos(\varphi)+i\cdot \sin(\varphi)
\end{equation}

"Mithilfe der Eulerschen Formel können wir jetzt eine beliebige komplexe Zahl umformen in die
\begin{equation}
	\textrm{Exponentialform:  } z=r(\cos \varphi + \sin \varphi)= r e^{i\varphi}
\end{equation}
mit $r=|z|$ und $\varphi=arg(z)$."\footnote{{Schmidt: Elementare Technomathematik. S. $299$.}}\\

Diese Darstellung von komplexen Zahlen hat ihre berühmteste Anwendung in der sogenannten \emph{eulerschen Identität}.
Wird nämlich in die eulersche Formel \eqref{euler} als Winkel $\varphi=\pi$ eingesetzt, so hat dieser Ausdruck einen erstaunlich simplen Wert.
Da der Kosinus beim Winkel $\pi$ den Wert $-1$ und der Sinus den Wert $0$ haben, entsteht die Formel

\begin{equation}
	\boxed{e^{i\pi}+1=0}
\end{equation}

\noindent welche von den allen Mathematikern als einzigartig angesehen wird.
Das Herausragende an ihr ist, dass sie "{}einen verblüffend einfachen Zusammenhalt zwischen vier der bedeutendsten mathematischen Konstanten herstellt:
Der Eulerschen Zahl $e$, der imaginären Einheit $i$, der Kreiszahl $\pi$ sowie der Einheit $1$ der reellen Zahlen."
\footnote{Engel/Fest: Komplexe Zahlen und ebene Geometrie. S. $32$.}
Der eulerschen Identität wird oftmals der Titel "{}schönste Formel der Mathematik"{} zugeschrieben und die Eleganz, mit der sie die wichtigsten Konstanten aus den verschiedenen Teilbereichen verbindet ist hoch geschätzt.\\



\section{Kreisteilungsgleichungen}



Komplexe Zahlen, in ihrer Exponentialform geschrieben, können eine Vielzahl von Problemen stark vereinfachen.
So wird zum Beispiel die Multiplikation von komplexen Zahlen viel übersichtlicher, wenn diese Darstellung benutzt wird.
Aus den Eigenschaften der Potenzfunktion entsteht somit der Zusammenhang
\[z_1\cdot z_2=(r_1 \cdot e^{i\varphi_1}) \cdot (r_2 \cdot e^{i\varphi_2})=r_1 r_2 \cdot e^{i(\varphi_1+\varphi_2)}\]
welcher es offensichtlich macht, dass Multiplikation von komplexen Zahlen durch eine Multiplikation der Beträge und Addition der Winkel dieser Zahlen erreicht wird.\footnote{Schmidt: Elementare Technomathematik. S. 299.}

Im Fall von Potenzieren von komplexen Zahlen gilt nach den Regeln der Multiplikation dann

\begin{equation}\label{potenz}
	z^n=r^n \cdot e^{i(n\varphi)}
\end{equation}



Wenn wir in diese Formel die eulersche Formel \eqref{euler} einsetzen, entsteht die Formel von Moivre:\footnote{Vgl. Bronstein/Mühlig/Musiol/Semendjajew: Taschenbuch der Mathematik. S. S. $39$.}
\begin{equation}
	[r(\cos\varphi+i\sin\varphi)]^n=r^n(\cos n\varphi+i\sin n\varphi)
\end{equation}

Einen sehr anschaulichen Effekt hat diese Formel nun wenn wir versuchen ihre geometrische Bedeutung zu Interpretieren.
Dafür benutzen wir sie zur Lösung der Gleichung $z^n=1$, einer sogenannten Kreisteilungsgleichung.
Wir bestimmen somit die $n$-te Wurzel aus $1$.

\begin{equation}\label{kreis}
	r^n(\cos n\varphi+i\sin n\varphi)=1
\end{equation}

"{}Zwei komplexe Zahlen sind gleich, wenn die zu ihrer Darstellung benötigte Vektoren gleich sind."
\footnote{Bronstein/Mühlig/Musiol/Semendjajew: Taschenbuch der Mathematik. S. S. $36$.}
Das heißt, dass in \eqref{kreis} die Beträge auf der linken und rechten Seite gleich sein müssen, während die Winkel gleich sind oder sich durch ein Vielfaches von $2\pi$ unterscheiden.\\


Die Zahl $1$ aus der rechten Seite von \eqref{kreis} hat den Betrag $1$ und da der Betrag $r$ die Länge eines Vektors beschreibt und somit stets eine positive reelle Zahl ist, hat die Gleichung $r^n=1$ in $\mathbb{R}$ nur eine Lösung $r=1$.

Der Winkel von $1$ in der Polarform ist gleich $0$, für den Wert des Winkels $\varphi$ aus der linken Seite von \eqref{kreis} gibt es jedoch mehrere Möglichkeiten.
Damit \eqref{kreis} wahr ist, muss $n\varphi$ gleich $0$ oder einem Vielfachen von $2\pi$ sein.
Es lässt sich leicht zeigen, dass es auf dem Einheitskreis genau $n$ solche Punkte gibt, welche beide Bedingungen für Betrag und Winkel erfüllen.
Ein solcher Punkt ist $z_0=(1,0)$ selber, damit wäre der Winkel $\varphi_0=0$, aber dies ist ja selbstverständlich.
Uns interessieren viel mehr die Punkte, welche komplexe Zahlen sind.
Der erste solche Punkt $z_1$, hat als $\varphi_1=\frac{2\pi}{n}$, da er durch $n$ Multiplikationen mit sich selbst nach einer Umdrehung wieder in $(1, 0)$ gelangen muss.
Nehmen wir für den nächsten Punkt $z_2$ den doppelten Winkel, so landet der Punkt nach dem Potenzieren mit $n$ mit insgesamt zwei Umdrehungen auf der $1$.
Dieses Verfahren lässt sich $n-1$ mal anwenden, um neue Punkte zu bekommen.
Versuchen wir nämlich weitere mögliche Winkel aus dem ersten Winkel $\varphi_1$ zu bilden, so müssen wir den Winkel $\varphi_1$ mit $n$ multiplizieren.
Nun ist aber $\varphi_1 \cdot n =\frac{2\pi}{n}\cdot n = 2\pi$, also eine ganze Umdrehung.
Damit ist dieser Punkt $z_n=(1,0)$ und würde nur den schon vorhandenen Punkt $z_0$ beschreiben.
Weiteres Addieren von $\varphi_1$ gibt auch keine neuen Wurzeln aus 1, da sie auch bloß vorherige Punkte ergeben.\footnote{Борис Трушин: Комплексные числа. Тригонометрическая форма. Формула Муавра.}
\\


Es liegen also $n$ verschiedene Punkte auf der Gaußschen Zahlenebene vor, die als $n$-te Wurzeln aus $1$ gesehen werden können.
Da die Winkel aller dieser Punkte Vielfache eines ersten Winkels $\varphi_1$ sind, lässt sich schließen, dass diese Lösungen in ihrer geometrischer Form in gleichmäßigen Abständen als Punkte auf dem Einheitskreis platziert sind.
Sie bilden somit die Ecken eines regelmäßigen Vielecks, welches als eine Ecke den Punkt $(1,0)$ hat.
Damit ist die graphische Darstellung beliebiger Wurzeln aus $1$, und damit der Lösungen von Kreisteilungsgleichungen, gefunden.

\begin{center}
	\definecolor{zzttqq}{rgb}{0.6,0.2,0}
	\definecolor{xdxdff}{rgb}{0.49019607843137253,0.49019607843137253,1}
	\definecolor{uuuuuu}{rgb}{0.26666666666666666,0.26666666666666666,0.26666666666666666}
	\begin{tikzpicture}[line cap=round,line join=round,>=triangle 45,x=1cm,y=1cm]
		\begin{axis}[
			x=2.5cm,y=2.5cm,
			axis lines=middle,
			ymajorgrids=true,
			xmajorgrids=true,
			xlabel={$Re(z)$},
			ylabel={$Im(z)$},
			xmin=-2.0741831415874628,
			xmax=2.074491179889266,
			ymin=-1.8844974113330106,
			ymax=1.5592395660879161,
			xtick={-2,-1.5,...,2},
			ytick={-2,-1.5,...,2},]
			\clip(-2.0741831415874628,-1.8844974113330106) rectangle (2.074491179889266,1.5592395660879161);
			\draw [line width=1pt] (0,0) circle (2.5cm);
			\draw [line width=1pt,color=zzttqq] (1,0)-- (0.30901699437494745,0.9510565162951535);
			\draw [line width=1pt,color=zzttqq] (0.30901699437494745,0.9510565162951535)-- (-0.8090169943749471,0.5877852522924731);
			\draw [line width=1pt,color=zzttqq] (-0.8090169943749471,0.5877852522924731)-- (-0.8090169943749472,-0.5877852522924729);
			\draw [line width=1pt,color=zzttqq] (-0.8090169943749472,-0.5877852522924729)-- (0.30901699437494723,-0.9510565162951534);
			\draw [line width=1pt,color=zzttqq] (0.30901699437494723,-0.9510565162951534)-- (1,0);
			\begin{scriptsize}
				\draw [fill=uuuuuu] (0,0) circle (2pt);
				\draw [fill=uuuuuu] (1,0) circle (2.5pt);
				\draw[color=uuuuuu] (1.1356361465438737,0.10214419464243102) node {\Large $z_0$};
				\draw [fill=uuuuuu] (0.30901699437494745,0.9510565162951535) circle (2.5pt);
				\draw[color=uuuuuu] (0.42756333225017244,1.123036732249006) node {\Large $z_1$};
				\draw [fill=uuuuuu] (-0.8090169943749471,0.5877852522924731) circle (2.5pt);
				\draw[color=uuuuuu] (-0.9022414439533022,0.7090298303635292) node {\Large $z_2$};
				\draw [fill=uuuuuu] (-0.8090169943749472,-0.5877852522924729) circle (2.5pt);
				\draw[color=uuuuuu] (-0.9022414439533022,-0.7081143470310872) node {\Large $z_3$};
				\draw [fill=uuuuuu] (0.30901699437494723,-0.9510565162951534) circle (2.5pt);
				\draw[color=uuuuuu] (0.42756333225017244,-1.1490152839400384) node {\Large $z_4$};
			\end{scriptsize}
		\end{axis}
	\end{tikzpicture}
	
	\begin{small}
		Abbildung $2$: Beispiel: Fünfte Wurzeln aus $1$
	\end{small}
\end{center}

Wollen wir die $n$-ten Wurzeln aus einer beliebigen komplexen Zahl $z=r\cdot e^i\varphi$ ziehen, wir das obengenannte $n$-Eck skaliert und gedreht.
Genauer gesagt:\\

\noindent\emph{"{}Eine komplexe Zahl $z=r e^{i\varphi}$ besitzt genau $n$ verschiedene Wurzeln}
\begin{equation}\label{vieleck}
	\sqrt[n]{z}=\sqrt[n]{r}\left(\cos\frac{\varphi+2k\pi}{n}+i\sin\frac{\varphi+2k\pi}{n}\right) \textrm{ für } k=0, 1, \dots, n-1
\end{equation}
\emph{In der [Gaußschen Zahlenebene] liegen die Wurzelwerte auf einem Kreis mit Radius $\sqrt[n]{r}$, und sie haben einen konstanten Winkelabstand $\frac{2\pi}{n}$ voneinander, beginnend mit $\frac{\varphi}{n}$. Sie Bilden folglich die Eckpunkte eines regelmäßigen $n$-Ecks."} \footnote{Schmidt: Elementare Technomathematik. S. $302$.}











\section{Beispiel der Anwendung}

Nun soll die Nützlichkeit der komplexen Zahlen an einer Aufgabe demonstriert werden, welche auf den ersten Blick sich gar nicht mit Algebra lösen lässt, da sie aus der Geometrie stammt. Dieses Problem ist sehr einfach formuliert, und könnte noch aus der Antike stammen, als große Mathematik mit Stöcken in Sand gezeichnet wurde. Die Aufgabe lautet:\\

\noindent \textit{In einen Einheitskreis wird ein regelmäßiges $n$-Eck eingezeichnet, sodass alle Ecken auf der Kreislinie liegen.
Eine der Ecken wird mit jeweils jeder anderen Ecke durch Strecken verbunden.
Was ist das Produkt der Längen dieser Strecken?}\footnote{Онлайн Уроки: Занятие 29 Комплексные числа в геометрии}\\


Aus dem 5. Abschnitt kennen wir Kreisteilungsgleichungen. Diese benötigt man zum Lösen der Aufgabe. Zuallererst jedoch muss die komplexe Zahlenebene festgelegt werden. 

Setzen wir den Ursprung unseres Koordinatensystems in den Mittelpunkt des Kreises und richten es so aus, dass die Achse der Realteile durch eine der Ecken durchläuft.
Jene Ecke mit den Koordinaten $(1,0)$ nennen wir $w_0$ und von dieser Ecke aus werden wir auch die Linien zu den anderen Ecken ziehen.
Wenn wir nun alle Ecken als komplexe Zahlen ansehen, so können wir die Kreisteilungsgleichungen und die Formel \eqref{vieleck} verwenden und stellen fest, dass die Ecken die Lösungen der Gleichung
\begin{equation}
	z^n=1
\end{equation}
sind, wobei $z$ eine $n$-te Wurzel von $1$ ist.
Diese restlichen Ecken nennen wir nun $w_1$, $w_2$,$\dots$,$w_{n-1}$, dabei ist $w_n = w_0 = 1$.
Unser Vieleck sieht nun folgendermaßen aus:

\begin{center}

\definecolor{zzttqq}{rgb}{0.6,0.2,0}
\definecolor{uuuuuu}{rgb}{0.26666666666666666,0.26666666666666666,0.26666666666666666}
\begin{tikzpicture}[line cap=round,line join=round,>=triangle 45,x=1cm,y=1cm]
	\begin{axis}[
		x=1cm,y=1cm,
		axis lines=middle,
		ymajorgrids=true,
		xmajorgrids=true,
		scale=3,
		xmin=-1.7,
		xmax=1.7,
		ymin=-1.7,
		ymax=1.7,
		xlabel={$Re(z)$},
		ylabel={$Im(z)$},
		xtick={-2,-1.5,...,2},
		ytick={-2,-1.5,...,2},
		ytick pos=right,
		yticklabel={\SI[round-mode=places, round-precision=1]{\tick}{}}
		]
		\clip(-2.038767629087051,-1.7478947202599961) rectangle (1.7928535328621873,1.7464358464472318);
		\fill[line width=3.2pt,color=zzttqq,fill=zzttqq,fill opacity=0.2] (1,0) -- (0.6234898018587336,0.7818314824680298) -- (-0.2225209339563141,0.9749279121818236) -- (-0.9009688679024188,0.4338837391175584) -- (-0.9009688679024189,-0.43388373911755773) -- (-0.22252093395631456,-0.9749279121818234) -- (0.6234898018587333,-0.7818314824680298) -- cycle;
		\draw [line width=0.5pt] (0,0) circle (3cm);
		\draw [line width=3.2pt,color=zzttqq] (1,0)-- (0.6234898018587336,0.7818314824680298);
		\draw [line width=3.2pt,color=zzttqq] (0.6234898018587336,0.7818314824680298)-- (-0.2225209339563141,0.9749279121818236);
		\draw [line width=3.2pt,color=zzttqq] (-0.2225209339563141,0.9749279121818236)-- (-0.9009688679024188,0.4338837391175584);
		\draw [line width=3.2pt,color=zzttqq] (-0.9009688679024188,0.4338837391175584)-- (-0.9009688679024189,-0.43388373911755773);
		\draw [line width=3.2pt,color=zzttqq] (-0.9009688679024189,-0.43388373911755773)-- (-0.22252093395631456,-0.9749279121818234);
		\draw [line width=3.2pt,color=zzttqq] (-0.22252093395631456,-0.9749279121818234)-- (0.6234898018587333,-0.7818314824680298);
		\draw [line width=3.2pt,color=zzttqq] (0.6234898018587333,-0.7818314824680298)-- (1,0);
		\draw [line width=2pt] (1,0)-- (0.6234898018587336,0.7818314824680298);
		\draw [line width=1.2pt] (1,0)-- (-0.2225209339563141,0.9749279121818236);
		\draw [line width=2pt] (1,0)-- (-0.9009688679024188,0.4338837391175584);
		\draw [line width=2pt] (1,0)-- (-0.9009688679024189,-0.43388373911755773);
		\draw [line width=2pt] (1,0)-- (-0.22252093395631456,-0.9749279121818234);
		\draw [line width=2pt] (1,0)-- (0.6234898018587333,-0.7818314824680298);
		\begin{scriptsize}
			\draw [fill=uuuuuu] (1,0) circle (2pt);
			\draw[color=uuuuuu] (1.15,0.10) node {\Large $w_0$};
			\draw [fill=uuuuuu] (0.6234898018587336,0.7818314824680298) circle (2pt);
			\draw[color=uuuuuu] (0.8,0.9) node {\Large $w_1$};
			\draw [fill=uuuuuu] (-0.2225209339563141,0.9749279121818236) circle (2pt);
			\draw[color=uuuuuu] (-0.19041516716083384,1.1) node {\Large $w_2$};
			\draw [fill=uuuuuu] (-0.9009688679024188,0.4338837391175584) circle (2pt);
			\draw[color=uuuuuu] (-0.95,0.5962749166719764) node {\Large $w_3$};
			\draw [fill=uuuuuu] (-0.9009688679024189,-0.43388373911755773) circle (2pt);
			\draw[color=uuuuuu] (-1,-0.5) node {\Large $w_4$};
			\draw [fill=uuuuuu] (-0.22252093395631456,-0.9749279121818234) circle (2pt);
			\draw[color=uuuuuu] (-0.3,-1.1) node {\Large $w_5$};
			\draw [fill=uuuuuu] (0.6234898018587333,-0.7818314824680298) circle (2pt);
			\draw[color=uuuuuu] (0.75,-0.9) node {\Large $w_6$};
		\end{scriptsize}
	\end{axis}
\end{tikzpicture}

\begin{small}
	Abbildung 3: Skizze zur Aufgabenstellung
\end{small}

\end{center}



Jeder Punkt in einem Koordinatensystem lässt sich durch dessen Ortsvektor, einen Vektor beginnend im Ursprung und endend in diesem Punkt, beschreiben. Komplexe Zahlen in ihrer geometrischer Darstellung werden genauso addieren und subtrahieren wie Vektoren und aus den Regeln der Subtraktion von Vektoren folgt, dass die Strecke, welche die Punkte $w_0$ und $w_k$ verbindet, gleich dem Vektor $\vec{v_k}=\vec{w_k}-\vec{w_0}$ ist.
Der Punkt $w_0$ hat die Koordinaten (1,0) und somit ist die Länge des Vektors $\vec{v_k}$ gleich dem Betrag der Differenz der komplexen Zahlen $w_k$ und $1$, also $|w_k-1|$.


In der Aufgabenstellung wird nach dem Produkt aller solcher Verbindungsstrecken gefragt.
Dieses Produkt, welches wir nun $X$ nennen, ist demnach gleich
\[X = |w_1-1|\cdot|w_2-1|\cdot\ldots\cdot|w_{n-1}-1|\]
Aus den Eigenschaften des Betrags folgt, dass sich $|w_k-1|$ durch $|1-w_k|$ ersetzten lässt.
Das gesamte Produkt entspricht dann
\[X=|1-w_1|\cdot|1-w_2|\cdot\ldots\cdot|1-w_{n-1}|\]
Nach den Regeln der Multiplikation von Beträgen ist das Produkt der Beträge gleich dem Betrag des Produkts. Damit gilt
\begin{equation}\label{X}
	X=|(1-w_1)\cdot(1-w_2)\cdot\ldots\cdot(1-w_{n-1})|
\end{equation}

An dieser Stelle lasst uns den Term, welcher innerhalb der Betragsstriche in $X$ liegt, ansehen.
Nennen wir diesen Term $H$. Damit ist
\begin{equation}\label{H}
	X=|H| \textrm{\quad wobei \quad}	H=(1-w_1)\cdot(1-w_2)\cdot\ldots\cdot(1-w_{n-1})
\end{equation}

Erinnern wir uns, dass die Eckpunkte des Vielecks auf dem Einheitskreis auch als Lösungen der Gleichung $z^n=1$ beziehungsweise $z^n-1=0$ betrachtet werden konnten. Die Folge \eqref{eq.faktor} des Fundamentalsatzes der Algebra besagt nun, dass 
\[z^n-1=(z-w_0)\cdot(z-w_1)\cdot(z-w_2)\cdot\ldots\cdot(z-w_{n-1})\]
da $w_0=1$ ist, schließt sich
\[z^n-1=(z-1)\cdot(z-w_1)\cdot(z-w_2)\cdot\ldots\cdot(z-w_{n-1})\]
Bei $z=1$ ist diese Gleichung nahezu identisch mit $H$ .
Der einzige Unterschied ist der erste Faktor $(z-1)$, welcher bei $H$ fehlt.
Möchten wir nun unsere Gleichung ohne diesen Faktor berechnen, so müssen wir sie durch $(z-1)$ teilen und es entsteht
\begin{equation}\label{bruch}
	H=(z-w_1)\cdot(z-w_2)\cdot\ldots\cdot(z-w_{n-1})=\frac{z^n-1}{z-1} \quad \textrm{ bei } z=1
\end{equation}
 Dies kann auch aufgeschrieben werden als
\[H=\frac{z^n-1}{z-1}=\frac{1-z^n}{1-z\hspace{2mm}}\]
Da wir versuchen, diese Gleichung bei $z=1$ zu lösen, müssen wir über einen Umweg gehen, ansonsten würde bei einem direkten Einsetzen der Nenner gleich $0$ sein.
Dabei fällt auf, dass $\frac{1-z^n}{1-z\hspace{1mm}}$ die Zusammenfassung einer geometrischen Reihe ist.
(Eine Geometrische Reihe $G_n$ ist ein Term der Form $a+aq+aq^2+\dots+aq^n$.)
Dass dies wahr ist lässt sich mithilfe einiger unkomplizierter Überlegungen zeigen:

"Wir werden beweisen, daß [sic!] für jeden Wert von $n$ [$\dots$]
\begin{equation}
	G_n=a+aq+aq^2+\dots+aq^n=a\frac{1-q^{n+1}}{1-q\hspace{6mm}}.
\end{equation} 
(Wir setzen $q\ne1$, lies: $q$ ungleich $1$, voraus, da sonst die rechte Seite von [der obenstehenden Behauptung] keinen Sinn hätte.)

Diese Behauptung ist für $n=1$ sicher gültig; denn dann besagt sie, daß [sic!]
\[G_1=a+aq=\frac{a(1-q^2)}{1-q}=\frac{a(1+q)(1-q)}{1-q}=a(1+q).\]
Wenn wir nun annehmen, daß [sic!]
\[G_r=a+aq+\dots+aq^r=a\frac{1-q^{r+1}}{1-q},\]
dann finden wir als Folgerung daraus
\begin{multline}
	G_{r+1}=(a+aq+\dots+aq^r)+aq^{r+1}=G_r+aq^{r+1}=a\frac{1-q^{r+1}}{1-q\hspace{6mm}}+aq^{r+1}\\
	=a\frac{(1-q^{r+1})+q^{r+1}(1-q)}{1-q}=a\frac{1-q^{r+1}+q^{r+1}-q^{r+2}}{1-q}=a\frac{1-q^{r+2}}{1-q\hspace{6mm}}
\end{multline}
Dies ist aber gerade die Behauptung [, die wir beweisen wollen,] für den Fall $n=r+1$. Damit ist der Beweis vollständig."\footnote{Courant/Robbins: Was ist Mathematik? S. $11$.}

Nun können wir $\frac{1-z^n}{1-z}$ als eine Geometrische Reihe aufschreiben.
\[\frac{1-z^n}{1-z\hspace{2mm}}=1+z+z^2+z^3+\dots+z^{n-1}\]
bedenken wir, dass $\frac{z^n-1}{z-1}=\frac{1-z^n}{1-z\hspace{1mm}}$ ist, gilt bei $z=1$
\[\frac{z^n-1}{z-1}=1+1+1^2+1^3+\dots+1^{n-1}=n\]
da die $1$ in dieser geometrischen Reihe insgesamt $n$ mal als Summand vorliegt.


Den Term $\frac{z^n-1}{z-1}$ haben wir ursprünglich aufgestellt, da er nach der Gleichung \eqref{bruch} äquivalent zu $H$ ist. Damit ist $H=n$ und aus \eqref{H} schließt sich
\[X=|n|=n\]
da $n$ eine natürliche Zahl ist\\

$X$ ist das Produkt der Längen aller Strecken, die eine Ecke des ursprünglichen Vielecks mit den restlichen verbinden. Somit ist die Aufgabe gelöst:\\

\noindent\emph{Das Produkt der Längen der "Diagonalen" {} eines auf dem Einheitskreis liegendes Vielecks ist gleich der Anzahl der Ecken im Vieleck.}






\newpage
\section{Literaturverzeichnis}




\renewcommand{\refname}{Printquellen}  % Name des Literaturverzeichnisses

Printquellen

\begin{thebibliography}{9}
%	\addcontentsline{toc}{section}{\refname}	% Literaturverzeichniss als Punkt in der Gliederung
	\bibitem{Zitat} Pieper, Herbert: Die komplexen Zahlen. Berlin $^3 1991$
	\bibitem{Zitat} Courant, Richard/ Robbins, Herbert: Was ist Mathematik? Berlin $^4 1962$
	\bibitem{Zitat} Engel, Joachim/ Fest, Andreas: Komplexe Zahlen und ebene Geometrie. Ludwigsburg $^3 2009$
	\bibitem{Zitat} Schmidt, Harald: Elementare Technomathematik. Amberg $2018$
	\bibitem{Zitat} Stewart, Ian: Welt-Formeln. Reinbeck bei Hamburg  $^7 2017$
	\bibitem{Zitat} Ansorge, Rainer/ Oberle, Hans J./ Rothe, Kai/ Sonar, Thomas: Mathematik für Ingenieure. Hamburg $^4 2010$
	\bibitem{Zitat} Bronstein, Ilja N./ Mühlig, Heiner/ Musiol, Gerhart/ Semendjajew, Konstantin A.: Taschenbuch der Mathematik. Haan-Gruiten $^{10} 2016$
\end{thebibliography}

\renewcommand{\refname}{Internetquellen}  % Name des Literaturverzeichnisses

\begin{thebibliography}{9}
	%	\addcontentsline{toc}{section}{\refname}	% Literaturverzeichniss als Punkt in der Gliederung
	\bibitem{Zitat} Wikipedia: Polynomial remainder theorem, \url{https://en.wikipedia.org/wiki/Polynomial_remainder_theorem} (10.11.2020 - 02.01 Uhr)
	\bibitem{Zitat} Mathehilfe: Exponentialform der komplexen Zahlen, \url{https://www.youtube.com/watch?v=AX3kDS0ChDg} (10.11.2020 - 02.12 Uhr)
	\bibitem{Zitat} Борис Трушин: Комплексные числа. Тригонометрическая форма. Формула Муавра, \url{https://www.youtube.com/watch?v=GGaZ5IJEjXw} (10.11.2020 - 02.37 Uhr)
	\bibitem{Zitat} Онлайн Уроки: Занятие 29 Комплексные числа в геометрии, \url{https://youtu.be/geEP9gfv5m4?t=6941} (10.11.2020 - 03.03 Uhr)
\end{thebibliography}

\end{document} % Dokumentende

