

\documentclass[a4paper,12pt]{article} % Dokunmentklasse

%%% Arbeit mit Sprache
\usepackage{cmap}					% Suche im PDF
\usepackage[T2A]{fontenc}			% Kodierung
\usepackage[utf8]{inputenc}			% Kodierung Ursprungstext
\usepackage[ngerman]{babel}	% Sprache

%%% Erweiterungen für Mathematik
\usepackage{amsmath,amsfonts,amssymb,amsthm,mathtools} % AMS
\usepackage{icomma} % Intelligent comma

%% Formelnummer
\mathtoolsset{showonlyrefs=true} % Zeigt die Nummer der Formel nur wenn mit \eqref{} auf sie im Text verwiesen wird.

%%% Seite
\usepackage{extsizes} % Mehr verschiedener Schriftgrößen
\usepackage{geometry} % Einstellung der Ränder
\geometry{top=25mm}
\geometry{bottom=25mm}
\geometry{left=35mm}
\geometry{right=25mm}

\usepackage{fancyhdr} % Kopf- und Fußzeile
\pagestyle{fancy}
\renewcommand{\headrulewidth}{0mm}  % Dicke der Linie unter der Kopfzeile
\lfoot{Unten Links}
\rfoot{Unten Rechts}
\rhead{Oben Rechts}
\chead{Oben Mitte}
\lhead{Oben Links}
% \cfoot{Unten Mitte} % falls nicht angegeben: Seitennummer

\usepackage{setspace} % Zeilenabstand
\onehalfspacing % Zeilenabstand 1.5
%\doublespacing % Zeilenabstand 2
%\singlespacing % Zeilenabstand 1

%% Schriften
\usepackage{euscript}
%\usepackage{times}
%\usepackage{mathptmx} %times
\usepackage{mathrsfs} % Schöne Matheschrift

%% Zeilenumbruch in Formeln
\newcommand*{\hm}[1]{#1\nobreak\discretionary{}
{\hbox{$\mathsurround=0pt #1$}}{}}

%%% Kopfzeile
\author{Jacob Cazacov}
\title{Komplexe Zahlen von algebraischer Sicht aus betrachtet}
\date{\today}

\usepackage{cite} % Bibliographie
%\usepackage[superscript]{cite} % Verweise in obere Zeilen
%\usepackage[nocompress]{cite} % 
\usepackage{csquotes} % Weiteres Werkzeug für Verweise


\usepackage{pgfplots}	%GeoGebra
\pgfplotsset{compat=1.15}
\usepackage{mathrsfs}
\usetikzlibrary{arrows}





\begin{document} % Dokumentbeginn

\maketitle

\newpage

\tableofcontents

\newpage

\section{Einleitung}

Die Einführung der komplexen Zahlen war eine der wichtigsten Errungenschaft der Mathematik.
Die Idee, dass man einen neuen Zahlenbereich benutzen könnte, welcher auf den ersten Blick keinerlei Realitätsbezug habe, wurde lange Zeit von Mathematikern aller Welt als sinnlos gesehen.
Erst als das entdeckerische Verlangen, alle Geheimnisse unserer Welt über Zahlen zu lüften, die Wissenschaftler an das Ende aller anschaulicher und einfacher Mathematik trieb, waren sie bereit, diesen Sprung zu wagen.
Und mit erstaunlichem Erfolg.
Denn Komplexe Zahlen bieten nicht nur einen Lösungsweg für Polynome höheren Grades, sondern können große Teile der Geometrie sehr einfach in rein algebraischer Form ausdrücken, mit welcher man viel bequemer umgehen kann.
Insgesamt bildet das Rechnen in $\mathbb{C}$, dem Zahlenbereich der komplexen Zahlen, ein sehr wichtiges Werkzeug für eine Vielzahl von Naturwissenschaften.

\section{Geschichte und Definition}

Der Zahlenbereich der reellen Zahlen war lange Zeit lang der größte und am meisten umfassendste Zahlenbereich, den die Menschheit kannte.
Er leistete auch sehr gute Dienste, denn mit ihm konnten nahezu alle algebraische Probleme gelöst werden und für Jahrtausende war dies genügend.
Umso provozierender war die Gleichung 
\[x^2+1=0\]
welche keine Lösung in den reellen Zahlen hatte, da es in ihnen keine Wurzel aus $-1$ gibt.
Das Problem ließ sich leicht umgehen, indem einfach festgelegt wurde, dass diese Gleichung keine Lösung habe, es war jedoch nicht aus der Welt geschafft.
Im Jahre 1545 veröffentlichte Geronimo Cardano (1501-1576) in seinem Werk \emph{"{Ars} magna de Regulis Algebraicis"} ein Verfahren, mit dem sich kubische Gleichungen der Form $ax^3+bx^2+cx+d=0$ lösen ließen.
Dieses Verfahren hatte jedoch ein Problem, als Zwischenergebnisse mussten teilweise Wurzeln aus negativen Zahlen gezogen werden.  
Im Laufe der Zeit stieß man immer öfter auf solche Schwierigkeiten in einer Vielzahl von verschiedenen Aufgaben und dies führte dazu, dass einige Mathematiker sich entschlossen das Problem direkt anzugehen.

Somit wurde die imaginäre Einheit $i$ als
\[i^2=-1\]
definiert, wobei für $i$ dieselben arithmetischen Regeln gelten, wie für alle anderen Zahlen.

Dieses $i$ eröffnet nun zahlreiche Möglichkeiten.
Zuallererst natürlich entstanden nun Zahlen der Form $z=a+ib$, die sogenannten komplexen Zahlen.
Sie konnten nun in allen möglichen Bereichen der Mathematik und sogar Physik angewandt werden.
$\mathbb{C}$ beschreibt ihre Gesamtheit und wurde somit zum neuen besten Zahlenbereich zum Lösen von Polynomen.


\section{Algebraische Abgeschlossenheit von $\mathbb{C}$}

Wie schon vorher erwähnt wurde, dienten die komplexen Zahlen ursprünglich als Mittel zur Lösung von Polynomen hohen Grades.
Da dies ihr wichtigster Verwendungszweck war, beschreibt auch einer der wichtigsten Sätze der Mathematik eben diese Eigenschaft.
Der sogenannte Fundamentalsatz der Algebra wurde 1799 von Carl Friedrich Gauß in seiner Dissertation bewiesen und lautet:\\


\noindent Jede Gleichung folgender Form (wobei $n \geq 1 $):
\begin{equation}\label{eq:funda}
\boxed{
a_nz^n + a_{n-1}z^{n-1} + a_{n-2}z^{n-2} + \dots + a_1z + a_0 = 0
}
\end{equation} 
besitzt \underline{mindestens eine} Lösung in $\mathbb{C}$.\\



Mit diesem Satz lassen sich jegliche Polynome in den komplexen Zahlen lösen und er hat somit eine grundlegende Bedeutung für die Mathematik.
Der Fundamentalsatz der Algebra wird manchmal auch in einer leicht verschiedenen, mehr umfassenden Form ausgedrückt, welche sich aus der ersten Formulierung einfach herausleiten lässt.\\

\noindent Man nehme ein beliebiges Polynom $f(x)$ der $n$-ten Ordnung und teile es durch den Term $(x-a)$.
\begin{equation}\label{eq:bezout1}
f(x) = (x - a) \times g(x) + r(x)
\end{equation} 
Dabei ist $a$ eine beliebige Zahl, $g(x)$ das entstehende Polynom der Ordnung $n-1$ und $r(x)$ der Rest, der aus dieser Teilung hervorgeht.
Da $f(x)$ durch ein Polynom ersten Grades geteilt wird, ist dieser Rest $r(x)$ eine Konstante.
Setzt man nun in die obenstehende Gleichung \eqref{eq:bezout1} ein $x=a$, dann folgt:
\[ f(a) = (a-a) \times g(a) + r(x) \]
Und da $(a - a) \times g(a) = 0 \times g(x) = 0 $ ist gilt:
\[ f(a) = r(a) \]
In Worten ausgedrückt:

\noindent \emph{Teilt man ein Polynom $f(x)$ durch den Term $(x-a)$, so ist der Rest, der in dieser Division entsteht, gleich dem Wert der Funktion $f(x)$ bei $x=a$.}\\

Diese Tatsache hat im Deutschen keinen eigenen Namen, ist jedoch im englischsprachigen Raum als das "Little Bézout's theorem" bekannt.

Sei nun $x_0$ eine Nullstelle des Polynoms $f(x)$, also $f(x_0)=0$. Dann folgt aus dem "Little Bézout's theorem" {}in diesem Spezialfall:


\begin{equation}\label{eq:bezout0}
r(x_0) = f(x_0)=0
\end{equation}

Demnach lässt sich ein beliebiges Polynom f(x) durch ein lineares Polynom $(x-x_0)$ mit einer Nullstelle als $x_0$ restlos Teilen.\\

Wenn nun ein Polynom $f(x)$ des $n$-ten Grades vorliegt, so besagt die erste Fassung des Fundamentalsatzes der Algebra \eqref{eq:funda}, dass eine Nullstelle $x_0$ existiert. Folglich lässt sich nun unser $f(x)$ mithilfe des "Little Bézout's theorem" {}umformen zu
\[f(x)=(x-x_0) \times g(x)\]
wobei sich f(x) restlos durch $(x-x_0)$ teilt und $g(x)$ ein Polynom $n-1$-ter Ordnung ist.
Dieses $g(x)$ hat jedoch nach dem Fundamentalsatz auch eine Nullstelle $x_1$.
Es lässt sich durch $(x-x_1)$ restlos teilen und damit folgt $f(x) = (x-x_0) \times (x-x_1) \times h(x)$.
Dieselbe Überlegungen können bei $h(x)$ angewandt werden und so weiter, wobei der Grad der Funktion in Term bei jedem Schritt sich um $1$ verkleinert.
Insgesamt lässt sich diese Methode auf dem Polynom $n$ mal verwenden und es teilt sich ohne Rest in $n$ Faktoren auf.\\

Hiermit kann jedes Polynom $n$-ten Grades in $n$ Faktoren zerteilt werden.
Dabei bildet es die Form
\[(x-x_0)(x-x_1)(x-x_2)\dots(x-x_{n-1})\]
mit den komplexen Zahlen $x_0, x_1, x_2,\dots,x_{n-1}$ als Nullstellen des Polynoms.\\




Als Folgerung hieraus kann nun eine zweite Fassung des Fundamentalsatzes der Algebra gebildet werden:\\

\noindent Jede Gleichung folgender Form (wobei $n \geq 1$):
\begin{equation}\label{eq:funda2}
	\boxed{	a_nz^n + a_{n-1}z^{n-1} + a_{n-2}z^{n-2} + \dots + a_1z + a_0 = 0
	}
\end{equation}
besitzt \underline{genau $n$} Lösungen in $\mathbb{C}$, wobei die Vielfachheit der Lösung beachtet werden muss.\\





Aus dieser Tatsache folgt direkt die algebraische Abgeschlossenheit von $\mathbb{C}$, nämlich dass jedes Polynom sich in diesem Zahlenbereich lösen lässt. 
Komplexe Zahlen wurden somit das Mittel schlechthin im Umgang mit Polynomen, sie endeten viele Probleme und eröffneten viele Möglichkeiten.
Diese Zahlen fanden Anwendung bei der Mathematik der Reste und hatte wegweisende Bedeutung für die Zahlentheorie.
Sie sind zweifellos die Grundlage für viele Teilbereiche der Mathematik. 



\section{Kreisteilungsgleichungen}

Die komplexen Zahlen sind jedoch nicht nur zur Lösung von Polynomen verwendbar, sondern werden auch bei Aufgaben benutzt, die auf den ersten Blick gar nicht mit traditioneller Algebra verbunden sind.
Um diese Aspekte zu erkennen, müssen wir jedoch zuerst ihre geometrische Darstellung betrachten.\\

Jede komplexe Zahl $z$ lässt sich wie wir wissen in der Form $z=a+bi$ schreiben, mit $a$ und $b$ als reelle Zahlen.
Hierbei besteht sie aus zwei Teilen.
Das $a$ wird als der "Realteil" $Re(z)$ der Zahl $z$ bezeichnet, dagegen ist $b$ der "{Imaginärteil}" $Im(z)$.
Geometrisch werden solche Zahlen auf leicht unerwartete Weise dargestellt, welche erst einmal erfunden werden musste.\\

Die reellen Zahlen sind offensichtliche ein Spezialfall der komplexen Zahlen, nämlich wenn der Imaginärteil gleich null ist.
Damit kann der Wert einer reellen Zahl als der Realteil einer komplexen Zahl $z$ mit $Im(z)=0$ angesehen werden.
Da alle reelle Zahlen als Punkte auf einer Zahlengerade visualisiert werden und $\mathbb{R}$ eine Teilmenge von $\mathbb{C}$ ist, muss auch in der geometrischen Form der komplexen Zahlen die reelle Zahlengerade vorhanden sein.
Der Imaginärteil wird dagegen in der vollkommen verschiedenen imaginären Einheit $i$ angegeben und lässt sich somit nicht auf dieser Zahlengerade darstellen.
Um diese beiden Bedingungen zu erfüllen wurde eine elegante Lösung gefunden\\


\emph{Bei der komplexen Zahlenebene, auch Gaußsche Zahlenebene genannt, handelt es sich um ein klassisches descartsches Koordinatensystem mit einer Achse, welche die Größe des Realteils $Re(z)$ der komplexen Zahl angibt, und der zweiten Achse für den Imaginärteil $Im(z)$. Jede komplexe Zahl wird als Punkt in dieser Ebene veranschaulicht.}\\

Somit ist die Realachse gleichzeitig auch die Zahlengerade von $\mathbb{R}$, da für jeden Punkt auf ihr gilt $Im(z)=0$.\\






% Komplexe Zahlenebene Bild
 



Wir können nun komplexe Zahlen als Punkte in einem Koordinatensystem betrachten.
Statt der klassischen Angabe der Position mit zwei Werten, einen für jede Achse, kann auch hier die Polarform benutzt werden.
In einem solchen Fall wäre bei, einem Winkel $\varphi$ und der Entfernung zum Ursprung $l$, der Realteil der Zahl gleich $Re(z)=l\times\cos(\varphi)$ und der Imaginärteil $Im(z)=l\times\sin(\varphi)$.
Jede komplexe Zahl ist also gleich $z=l\times(\cos(\varphi)+i\sin(\varphi))$

Betrachten wir nun den Sinus, beziehungsweise den Kosinus.
Wollen wir sie mit anderen Funktionen vergleichen, so wäre es besser wenn sie in der Polynomform vorlägen, da mit dieser in der Algebra leicht umgegangen werden kann.
Um dies zu erreichen benötigen wir das Taylorpolynom.
Dieses Polynom ist perfekt dazu geeignet komplizierte Funktionen mit hoher Genauigkeit anzunähern.
Das Polynom lässt sich nur dann anwenden, wenn die anzunähernde Funktion sich zureichend oft ableiten lässt, am besten unbeschränkt oft.
Für sowohl den Sinus als auch den Kosinus ist letzteres wahr.
Bei wiederholter Ableitung laufen sie durch die Werte $\sin(x)$, $\cos(x)$, $-\sin(x)$ und $-\cos(x)$ unendlich oft durch.
Das Taylorpolynom kann mit einer angegebenen Entwicklungsstelle $x_0$ den Wert jeder Funktion $f(x)$ an der Position $x$ durch die Formel
\begin{equation}
	T_n(x)=\frac{f(x_0)}{0!}\times(x-x_0)^0 + \frac{f^{(1)}(x_0)}{1!}\times(x-x_0)^1 +% \frac{f^{(2)}(x_0)}{2!}\times(x-x_0)^2 +
	\dots +\frac{f^{(n)}(x_0)}{n!}\times(x-x_0)^n
\end{equation} %Video von Mathepeter
annähern, wobei bei steigender Anzahl $n$ der Polynomglieder die Genauigkeit der Annäherung steigt.

Versuchen wir nun den $cos(x)$ durch ein unendlich langes Taylorpolynom, eine sogenannte Taylorreihe, so sehr anzunähern, dass das Polynom getrost der Funktion gleichgesetzt werden kann.
Wir bestimmen $x_0=0$, da an dieser Stelle die Sinus- und Kosinuswerte einfach bestimmt werden können.
\[cos(x)=\frac{cos(0)}{0!}\times(x-0) + \frac{-sin(0)}{1!}\times(x-0) + \frac{-cos(0)}{2!}\times(x-0)^2+\frac{sin(0)}{3!}\times(x-0)^3+\dots\]
Dabei fällt auf, dass bei jedem zweiten Polynomglied im Zähler des Bruches ein positives oder negatives $sin(0)$ vorliegt.
Da $sin(0)=0$ ist, haben diese Glieder keinen Einfluss auf den Wert des Polynoms und können weggelassen werden. In den restlichen Gliedern ligt ein $cos(x)$ vor. Dies ist bei $x=0$ gleich $cos(0)=1$. Damit lässt sich das Polynom vereinfachen zu
\[cos(x)=1-\frac{x^2}{2!}+\frac{x^4}{4!}-\frac{x^6}{6!}+\frac{x^8}{8!}+\dots\]
Die Polynomformen anderer Funktionen erhält man nach analogem Verfahren, wie zum Beispiel die Sinusfunktion
\[sin(x)=x-\frac{x}{1!}+\frac{x^3}{3!}-\frac{x^5}{5!}+\frac{x^7}{7!}+\dots\]
oder hier die $e$-Funktion
\[e^x=1+\frac{x}{1!}+\frac{x^2}{2!}+\frac{x^3}{3!}+\frac{x^4}{4!}+\dots\]

Mithilfe des Sinus und Kosinus können wir nun komplexe Zahlen als Polynome aufschreiben. Sehen wir uns der Einfachheit halber zuerst eine Zahl an, die auf dem Einheitskreis liegt und damit als $l=1$ hat.
\begin{equation}
	z=%\times(\cos(\varphi)+i\times\sin(\varphi))=
	\left(1-\frac{x^2}{2!}+\frac{x^4}{4!}-\frac{x^6}{6!}+\dots\right)+i\left(x-\frac{x}{1!}+\frac{x^3}{3!}-\frac{x^5}{5!}+\dots\right)
\end{equation}
Vergleicht man nun $z$ mit der Polynomform von $e^{i\varphi}$ stellt man fest, dass sie gleich sind. 
In der Taylorreihe von $e^{i\varphi}$ haben die Polynomglieder im Zähler ein $i^n\varphi^n$. Wir müssen nun bedenken, dass $i^2=-1$ ist und durch Umformungen und Umgruppierungen entsteht
\begin{multline}	%Wahrscheinlich mit Fehler
	e^{i\varphi}=1+\frac{i\varphi}{1!}+\frac{i^2\varphi^2}{2!}+\frac{i^3\varphi^3}{3!}+\frac{i^4\varphi^4}{4!}+\dots=
	1+\frac{i\varphi}{1!}-\frac{\varphi^2}{2!}-\frac{i\varphi^3}{3!}+\frac{\varphi^4}{4!}+\dots=\\
	\left(1-\frac{\varphi^2}{2!}+\frac{\varphi^4}{4!}-\frac{\varphi^6}{6!}+\dots\right)+i\times\left(\varphi-\frac{\varphi}{1!}+\frac{x\varphi^3}{3!}-\frac{\varphi^5}{5!}+\dots\right) =\\
	\cos(\varphi) + i\times\sin(\varphi)=z
\end{multline}
Hierbei handelt es sich um die berühmte \emph{"{}Eulersche Formel"}, welche Besagt, dass
\begin{equation} \label{euler}
	e^{i\varphi}=\cos(\varphi)+i\times\sin(\varphi)
\end{equation}
Daraus folgt, dass komplexe Zahlen auf dem Einheitskreis durch den Term $e^{i\varphi}$ dargestellt werden können. Für andere komplexe Zahlen muss noch der Abstand $l$ vom Ursprung bedacht werden und für sie gilt
\[z=l\times e^{i\varphi}\]


%Bis hier korrigiert

Diese Darstellung von komplexen Zahlen hat ihre berühmteste Anwendung in der sogenannten \emph{eulerschen Identität}.
Wird nämlich in die eulersche Formel \eqref{euler} als Winkel $\varphi=\pi$ eingesetzt, so hat dieser Ausdruck einen erstaunlich simplen Wert.
Der Kosinus hat beim Winkel $\pi$ den Wert $0$ und der Sinus den Wert $-1$.
Dadurch entsteht die Formel
\[e^{i\pi}+1=0\]
welche von den meisten Mathematikern als einzigartig angesehen wird.
Das Herausragende an ihr ist, dass sie "{}einen verblüffend einfachen Zusammenhalt zwischen vier der bedeutendsten mathematischen Konstanten herstellt:
Der Eulerschen Zahl $e$, der imaginären Einheit $i$, der Kreiszahl $\pi$ sowie der Einheit $1$ der reellen Zahlen." (Zitat)
Der eulerschen Identität wird oftmals der Titel "{}schönste Formel der Mathematik" zugeschrieben und die Eleganz, mit der sie die wichtigsten Konstanten aus den verschiedenen Teilbereichen der Mathematik verbindet, wird somit wertgeschätzt.\\




Komplexe Zahlen, in ihrer Potenzform geschrieben, können eine Vielzahl von Problemen stark vereinfachen.
So wird zum Beispiel die Multiplikation von komplexen Zahlen viel anschaulicher, wenn man diese Darstellung benutzt.
Auf recht einfache Weise entsteht somit der Zusammenhang
\[z_1\times z_2=(l_1 \times e^{i\varphi_1}) \times (l_2 \times e^{i\varphi_2})=l_1 l_2 \times e^{i(\varphi_1+\varphi_2)}\]
welcher es offensichtlich macht, dass Multiplikation von komplexen Zahlen durch eine Multiplikation der Längen und Addition der Winkel dieser Zahlen erreicht wird.

Im Fall von Potenzieren von komplexen Zahlen gilt nach den Regeln der Multiplikation dann
\[z^n=l^n \times e^{i(n\varphi)}\]

Einen sehr anschaulichen Effekt hat diese Formel nun wenn wir versuchen ihre geometrische Bedeutung zu Interpretieren.
Sie kann nämlich als die Formel für die Lösungen einer Gleichung der Form $z^n=h$ gesehen werden.
Setzen wir als $h=1$ ein, so erhalten wir eine Kreisteilungsgleichung $z^n=1$. Wir bestimmen in diesem Fall die $n$-te Wurzel aus $1$.
Benutzen wir nun die Formel $l^n \times e^{i(n\varphi)}=1$, entspricht $l$ für jedes $z$ gleich $l=1$, da $l^n=1$ gilt und $l$ nur positive Werte annehmen kann.

für den Wert des Winkels $\varphi$ gibt es jedoch mehrere Möglichkeiten.
Damit $e^{in\varphi}$ gleich $1$ ist, muss die Polarform des Terms als Winkel $0$ oder ein Vielfaches von $2\pi$ haben.
Dadurch fällt der Winkel multipliziert mit $n$ wieder zurück auf die Realachse zum Punkt $(1,0)$.
Es lässt sich leicht zeigen, dass es auf dem Einheitskreis genau $n$ solche Punkte gibt, welche beide Bedingungen für Länge und Winkel erfüllen.
Wenn wir nämlich den ersten solchen Punkt $z_1$ ansehen, so hat er als $\varphi=\frac{2\pi}{n}$, um nach $n$ Multiplikationen mit sich selbst nach einer Umdrehung wieder in $1$ zu gelangen.
Nehmen wir für den nächsten Punkt $z_2$ den doppelten Winkel, so landet der Punkt nach dem Potenzieren mit $n$ mit insgesamt zwei Umdrehungen auf der $1$.
Der Fundamentalsatz der Algebra besagt, dass es $n$ Lösungen für die Gleichung $e^{i(n\varphi)}=1$ gibt und wenn man dem obengenannten Verfahren weiter folgt, schließt sich, dass diese Lösungen in ihrer geometrischer Form in gleichmäßigen Abständen als Punkte auf dem Einheitskreis platziert sind.
Sie bilden somit die Ecken eines regelmäßigen Vielecks, welches als eine Ecke den Punkt $(1,0)$ hat.
Damit ist die graphische Darstellung beliebiger Wurzeln aus $1$ gefunden.





\section{Beispiel der Anwendung}

Nun soll die Nützlichkeit der komplexen Zahlen an einer Aufgabe demonstriert werden, welche auf den Ersten Blick sich gar nicht mit Algebra lösen lässt, da sie aus der Geometrie stammt. Dieses Problem ist sehr einfach formuliert, und könnte noch aus der Antike stammen, als große Mathematik mit Stöcken in Sand gezeichnet wurde. Die Aufgabe Lautet:\\

\noindent \textit{In einen Einheitskreis wird ein regelmäßiges $n$-Eck eingezeichnet, sodass alle Ecken auf der Kreislinie liegen.
Eine der Ecken wird mit jeweils jeder anderen Ecke durch Strecken verbunden.
Was ist das Produkt der Längen dieser Strecken?
}\\

Aus dem Abschnitt (4...) kennen wir Kreisteilungsgleichungen. Diese benötigt man zum Lösen der Aufgabe. Zuallererst jedoch muss die komplexe Zahlenebene festgelegt werden. 

Setzen wir den Ursprung unsres Koordinatensystems in den Mittelpunkt des Kreises und richten es so aus, dass die Achse der Realanteile durch eine der Ecken durchläuft sodass diese Ecke nun die Koordinaten $(1,0)$ hat.
Jene Ecke nennen wir $w_0$ und von dieser Ecke aus werden wir auch später die Linien zu den anderen Ecken ziehen.\\

Wenn wir nun alle Ecken als komplexe Zahlen ansehen, so können wir die Kreisteilungsgleichungen verwenden und kommen zum Schluss, dass
\begin{equation}
	z^n=1
\end{equation}

wobei $z$ eine $n$-te Wurzel von $1$ ist und somit auch ein Eckpunkt unseres $n$-Ecks.
Diese restlichen Ecken nennen wir nun $w_1$, $w_2$,$\dots$,$w_{n-1}$, dabei ist $w_n = w_0 = 1$.\\ 

Jeder Punkt in einem Koordinatensystem lässt sich durch einen Vektor beginnend im Ursprung und endend in diesem Punkt beschreiben. Komplexe Zahlen in ihrer geometrischer Darstellung werden genauso addieren und subtrahieren wie Vektoren und aus den Regeln der Subtraktion von Vektoren folgt, dass ein Vektor $\vec{v}$, welcher in $w_0$ beginnt und in $w_z$ endet, gleich $\vec{v}=w_z-w_0$ ist.
$w_0$ hat die Koordinaten (1,0).
Damit ist die Länge des Vektors, und somit der Strecke, welcher $w_0$ und $w_z$ verbindet gleich $|\vec{v}| = w_z - 1$.

In der Aufgabenstellung wird nach dem Produkt aller Verbindungsstrecken gefragt.
Dieses Produkt, welches wir nun $X$ nennen, ist demnach gleich
\[X = |(w_1-1)|\times|(w_2-1)|\times\dots\times|(w_{n-1}-1)|\]
Jeder Faktor dieses Produkts ist der Betrag der jeweiligen Klammer, da die Längen der Strecken nur positive Werte annehmen können.
Das bedeutet aber auch, dass man in jeder Klammer statt $(w_z-1)$ auch $(1-w_z)$ schreiben kann, ohne den Betrag zu verändern.
Das gesamte Produkt ist dann gleich 
\[X=|(1-w_1)|\times|(1-w_2)|\times\dots\times|(1-w_{n-1})|\]
Nach den Regeln der Multiplikation von Beträgen ist das Produkt der Beträge von Faktoren gleich dem Betrag des Proddukts dieser Faktoren. Also gilt
\begin{equation}\label{X}
	X=|(1-w_1)\times(1-w_2)\times\dots(1-w_{n-1})|
\end{equation}

An dieser Stelle lasst uns den Term, welcher sich innerhalb der Betragsstriche in $X$ liegt ansehen.
Nennen wir diesen Term $H$. Damit ist $X=|H|$ und
\begin{equation}\label{H}
	H=(1-w_1)\times(1-w_2)\times\dots(1-w_{n-1})
\end{equation}

Erinnern wir uns, dass die Eckpunkte des Vielecks auf dem Einheitskreis auch als Lösungen der Gleichung $z^n=1$ beziehungsweise $z^n-1=0$ dargestellt werden können. Die erweiterte Fassung des Fundamentalsatzes der Algebra (\ref{eq:funda2}) besagt nun, dass
\[z^n-1=(z-w_0)\times(z-w_1)\times(z-w_2)\times\dots(z-w_{n-1})\]
da $w_0=1$ ist, folgt
\[z^n-1=(z-1)\times(z-w_1)\times(z-w_2)\times\dots(z-w_{n-1})\]
Diese Gleichung ist nahezu identisch mit $H$ bei $z=1$.
Der einzige Unterschied ist der Faktor $(z-1)$, welcher bei $H$ fehlt.
Möchten wir nun unsere Gleichung ohne diesen Faktor berechnen, so müssen wir sie durch $(z-1)$ teilen und es entsteht
\begin{equation}\label{frac}
	H=(z-w_1)\times(z-w_2)\times\dots(z-w_{n-1})=\frac{z^n-1}{z-1}
\end{equation}
bei $z=1$. Dies kann auch aufgeschrieben werden als
\[\frac{z^n-1}{z-1}=\frac{1-z^n}{1-z\hspace{2mm}}\]
Da wir versuchen, diese Gleichung bei $z=1$ zu lösen, müssen wir über einen Umweg gehen, ansonsten würde bei einem direkten Einsetzen der Nenner gleich $0$ sein.
Dabei fällt auf, dass $\frac{1-z^n}{1-z\hspace{1mm}}$ die Zusammenfassung einer geometrischen Reihe ist.
(Eine Geometrische Reihe $G_n$ ist ein Term der Form $a+aq+aq^2+\dots+aq^n$.)
Dass dies wahr ist lässt sich mithilfe der vollständigen Induktion zeigen.

"Wir werden beweisen, daß [sic!] für jeden Wert von $n$ [$\dots$]
\begin{equation}
	G_n=a+aq+aq^2+\dots+aq^n=a\frac{1-q^{n+1}}{1-q\hspace{6mm}}.
\end{equation} 
(Wir setzen $q\ne1$, lies: $q$ ungleich $1$, voraus, da sonst die rechte Seite von [der obenstehenden Behauptung] keinen Sinn hätte.)

Diese Behauptung ist für $n=1$ sicher gültig; denn dann besagt sie, daß [sic!]
\[G_1=a+aq=\frac{a(1-q^2)}{1-q}=\frac{a(1+q)(1-q)}{1-q}=a(1+q).\]
Wenn wir nun annehmen, daß [sic!]
\[G_r=a+aq+\dots+aq^r=a\frac{1-q^{r+1}}{1-q},\]
dann finden wir als Folgerung daraus
\begin{multline}
	G_{r+1}=(a+aq+\dots+aq^r)+aq^{r+1}=G_r+aq^{r+1}=a\frac{1-q^{r+1}}{1-q\hspace{6mm}}+aq^{r+1}\\
	=a\frac{(1-q^{r+1})+q^{r+1}(1-q)}{1-q}=a\frac{1-q^{r+1}+q^{r+1}-q^{r+2}}{1-q}=a\frac{1-q^{r+2}}{1-q\hspace{6mm}}
\end{multline}
Dies ist aber gerade die Behauptung [, die wir beweisen wollen,] für den Fall $n=r+1$. Damit ist der Beweis vollständig."(Zitat-Zeug)

Nun können wir $\frac{1-z^n}{1-z}$ als eine Geometrische Reihe aufschreiben.
\[1\times\frac{1-z^n}{1-z\hspace{2mm}}=1+z+z^2+\dots+z^{n-1}\]
bedenken wir, dass $\frac{z^n-1}{z-1}=\frac{1-z^n}{1-z\hspace{1mm}}$ ist, gilt bei $z=1$
\[\frac{z^n-1}{z-1}=1+1+1^2+\dots+1^{n-1}=n\]
da die $1$ in dieser geometrischen Reihe insgesamt $n$ mal vorkommt.



% Optional: Zweiter Beweis: Polynomdivision



Den Term $\frac{z^n-1}{z-1}$ haben wir ursprünglich aufgestellt, da er nach der Gleichung (\ref{H}) äquivalent zu $H$ ist. Damit ist
\[H=n\]
$H$ selber war der Term innerhalb der Betragsstriche von $X$, denn $X=|H|$.
Da nun $H=n$ ist mit $n$ als eine zwingend positive Zahl, lässt sich $H$ mit $X$ direkt gleichstellen und man schließt heraus, dass:
\[X=n\]
$X$ ist das Produkt aller Längen der Strecken, die eine Ecke des ursprünglichen Vielecks mit den verbinden. Somit ist die Aufgabe gelöst und jenes Produkt ist gleich der Anzahl der Ecken im Vieleck.







\section{Literaturverzeichnis}

\renewcommand{\refname}{Literaturverzeichnis}  % Name des Literaturverzeichnisses

\begin{thebibliography}{9}
%	\addcontentsline{toc}{section}{\refname}	% Literaturverzeichniss als Punkt in der Gliederung
	\bibitem{Zitat} Quelle
	
\end{thebibliography}




\end{document} % Dokumentende

