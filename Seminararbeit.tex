

\documentclass[a4paper,12pt]{article} % Dokunmentklasse

%%% Arbeit mit Sprache
\usepackage{cmap}					% Suche im PDF
\usepackage[T2A]{fontenc}			% Kodierung
\usepackage[utf8]{inputenc}			% Kodierung Ursprungstext
\usepackage[ngerman]{babel}	% Sprache

%%% Erweiterungen für Mathematik
\usepackage{amsmath,amsfonts,amssymb,amsthm,mathtools} % AMS
\usepackage{icomma} % Intelligent comma

%% Formelnummer
\mathtoolsset{showonlyrefs=true} % Zeigt die Nummer der Formel nur wenn mit \eqref{} auf sie im Text verwiesen wird.

%%% Seite
\usepackage{extsizes} % Mehr verschiedener Schriftgrößen
\usepackage{geometry} % Einstellung der Ränder
\geometry{top=25mm}
\geometry{bottom=25mm}
\geometry{left=35mm}
\geometry{right=25mm}

\usepackage{fancyhdr} % Kopf- und Fußzeile
\pagestyle{fancy}
\renewcommand{\headrulewidth}{0mm}  % Dicke der Linie unter der Kopfzeile
\lfoot{Unten Links}
\rfoot{Unten Rechts}
\rhead{Oben Rechts}
\chead{Oben Mitte}
\lhead{Oben Links}
% \cfoot{Unten Mitte} % falls nicht angegeben: Seitennummer

\usepackage{setspace} % Zeilenabstand
\onehalfspacing % Zeilenabstand 1.5
%\doublespacing % Zeilenabstand 2
%\singlespacing % Zeilenabstand 1

%% Schriften
\usepackage{euscript}
%\usepackage{times}
%\usepackage{mathptmx} %times
\usepackage{mathrsfs} % Schöne Matheschrift

%% Перенос знаков в формулах (по Львовскому)
\newcommand*{\hm}[1]{#1\nobreak\discretionary{}
{\hbox{$\mathsurround=0pt #1$}}{}}

%%% Заголовок
\author{Jacob Cazacov}
\title{Komplexe Zahlen von algebraischer Sicht aus betrachtet}
\date{\today}





\begin{document} % конец преамбулы, начало документа

\maketitle

\newpage

\tableofcontents

\newpage

\section{Einleitung}

Die Einführung der komplexen Zahlen war eine äußerst wichtige Errungenschaft, der Mathematik. Die Idee, dass man einen neuen Zahlenbereich einführen könnte, welcher auf den ersten Blick keinerlei Realitätsbezug hat, wurde lange Zeit von Mathematikern aller Welt als sinnlos gesehen. Erst als das entdeckerische Verlangen, alle Geheimnisse unserer Welt über Zahlen zu lüften, die Wissenschaftler an das Ende aller anschaulicher und einfacher Mathematik trieb, waren sie bereit, diesen Sprung zu wagen. Und mit erstaunlichem Erfolg. Denn Komplexe Zahlen bieten nicht nur einen Lösungsweg für Polynome höheren Grades, sondern können große Teile der Geometrie sehr einfach in rein algebraischer Form ausdrücken, mit welcher man viel bequemer rechnen kann (Beleg erforderlich). Insgesamt bildet das Rechnen in $\mathbb{C}$ ein sehr wichtiges Werkzeug für eine Vielzahl von Naturwissenschaften.

\section{Algebraische Abgeschlossenheit von $\mathbb{C}$}

Wie schon vorher erwähnt wurde, dienten die komplexen Zahlen ursprünglich als Mittel zur Lösung von Polynomen hohen Grades.
Da dies ihr wichtigster Verwendungszweck war, beschreibt auch einer der wichtigsten Sätze der Mathematik eben diese Eigenschaft.
Der sogenannte Fundamentalsatz der Algebra wurde 1799 von Carl Friedrich Gauß in seiner Dissertation bewiesen und lautet:\\


\noindent Jede Gleichung der Form (wobei $n \geq 1 $):
\begin{equation}\label{eq:funda}
\boxed{
a_nz^n + a_{n-1}z^{n-1} + a_{n-2}z^{n-2} + \dots + a_1z + a_0 = 0
}
\end{equation} 
besitzt \underline{mindestens eine} Lösung in $\mathbb{C}$.



Mit diesem Satz lassen sich jegliche Gleichungen in den komplexen Zahlen lösen und er hat somit eine grundlegende Bedeutung für die Algebra.
Der Fundamentalsatz der Algebra wird manchmal auch in einer leicht verschiedenen, mehr umfassenden Form ausgedrückt, welche sich aus der ersten Formulierung herausleiten lässt.\\

\noindent Man nehme ein beliebiges Polynom $f(x)$ der $n$-ten Ordnung und teile es durch den Term $(x-a)$.
\begin{equation}\label{eq:bezout1}
f(x) = (x - a) \times g(x) + r(x)
\end{equation} 
Dabei ist $a$ eine beliebige Zahl, $g(x)$ ein Polynom der Ordnung $n-1$ und $r(x)$ der Rest, der aus dieser Teilung hervorgeht.
Da $f(x)$ durch ein Polynom ersten Grades geteilt wird, ist dieser Rest $r(x)$ eine Konstante.
Setzt man nun in die obenstehende Gleichung \eqref{eq:bezout1} ein $x=a$, dann folgt:
\[ f(a) = (a-a) \times g(a) + r(x) \]
Und da $(a - a) \times f(a) = 0 \times g(x) = 0 $ ist gilt:
\[ f(a) = r(a) \]
In Worten ausgedrückt:

\noindent \emph{Teilt man ein Polynom $f(x)$ durch den Term $(x-a)$, so ist der Rest dieser Division gleich dem Wert der Funktion $f(x)$ bei $x=a$.}

\noindent Diese Tatsache hat im Deutschen keinen eigenen Namen, ist jedoch im englischsprachigen Raum als das "Little Bézout's theorem" bekannt.

\noindent Da die Nullstelle eines Polynoms per Definition ein $x_0$-Wert ist, bei dem ein beliebiges Polynom $f(x_0) = 0$ ist, folgt aus dem "Little Bézout's theorem":
\begin{equation}\label{eq:bezout0}
r(x_0) = f(x_0)=0
\end{equation}
Demnach lässt sich jedes Polynom durch das Linearpolynom $(x-x_0)$ restlos teilen.\\

Wenn wir nun ein Polynom $f(x)$ des $n$-ten Grades haben, so können wir dank der ersten Fassung des Fundamentalsatzes der Algebra eine Nullstelle $x_0$ bestimmen. Demzufolge lässt sich nun unser $f(x)$ mithilfe des "Little Bézout's theorem" umformen zu
\[(x-x_0) \times g(x)\]
wobei sich f(x) restlos durch $(x-x_0)$ teilt und $g(x)$ ein Polynom $n-1$-ter Ordnung ist.
Dieses $g(x)$ hat jedoch auch eine Nullstelle $x_1$ nach dem Fundamentalsatz und es lässt sich auch durch $(x-x_1)$ restlos teilen und zu $f(x) = (x-x_0) \times (x-x_1) \times h(x)$ werden. Dieselbe Überlegungen können zu $h(x)$ angewandt werden und so weiter, wobei der Grad der Restfunktion bei jedem Schritt sich um $1$ verkleinert.
Insgesamt lässt sich diese Methode auf dem Polynom $n$ mal verwenden und es teilt sich ohne Rest in $n$ Faktoren auf. Daraus folgt, dass $f(x)$ $n$ Nullstellen besitzt.

\noindent Nun lässt sich eine weitere Fassung des Fundamentalsatzes formulieren:\\

\noindent Jede Gleichung der Form (wobei $n \geq 1$):
\begin{equation}\label{eq:funda2}
	\boxed{	a_nz^n + a_{n-1}z^{n-1} + a_{n-2}z^{n-2} + \dots + a_1z + a_0 = 0
	}
\end{equation}
besitzt \emph{genau $n$} Lösungen in $\mathbb{C}$, wobei die Vielfachheit der Lösung beachtet werden muss.


\section{Kreisteilungsgleichungen}

Die komplexen Zahlen sind jedoch nicht nur zur Lösung von Polynomen verwendbar, sondern werden auch bei Aufgaben benutzt, die auf den ersten Blick gar nicht mit traditioneller Algebra verbunden sind.
Um diese Aspekte zu erkennen, müssen wir jedoch zuerst ihre geometrische Darstellung betrachten.

Jede komplexe Zahl $z$ lässt sich in der Form $z=a+bi$ schreiben, mit $a$ und $b$ als reelle Zahlen.
Hierbei besteht sie aus zwei Teilen.
Das $a$ wird als der "Realteil" $Re(z)$ der Zahl $z$ bezeichnet, dagegen ist $b$ der "{Imaginärteil}" $Im(z)$.
Geometrisch werden solche Zahlen leicht unerwartet dargestellt. Die reellen Zahlen sind offensichtlicher-weise ein Spezialfall der komplexen Zahlen, nämlich wenn der Imaginärteil gleich null ist.
Da alle reelle Zahlen in einer Zahlengerade visualisiert werden und $\mathbb{R}$ eine Teilmenge von $\mathbb{C}$ ist, muss auch in der geometrischen Form diese Zahlengerade ein Teil der Darstellung der Komplexen Zahlen sein.
Dies wird elegant gelöst.

Bei der komplexen Zahlenebene $\mathbb{C}$ handelt es sich um ein klassisches descartsches Koordinatensystem mit einer Achse, welche die Größe des Realteils der komplexen Zahl angibt, und der zweiten Achse für den Imaginärteil. Somit ist die Realachse gleichzeitig auch die Zahlengerade von $\mathbb{R}$, da für jeden Punkt auf ihr gilt $Im(z)=0$.

Wir können nun komplexe Zahlen als Punkte in einem Koordinatensystem betrachten.
Statt der Klassischen Angabe der Position mit zwei Werten, einen für jede Achse, kann auch hier die Polarform benutzt werden.
In einem solchen Fall wäre bei, einem Winkel $\varphi$ und der Entfernung zum Ursprung $l$, der Realteil der Zahl gleich $Re(z)=l\times\cos(\varphi)$ und der Imaginärteil $Im(z)=l\times\sin(\varphi)$.
Jede komplexe Zahl ist also gleich $z=l\times(\cos(\varphi)+\sin(\varphi))$

Betrachten wir nun den Sinus, beziehungsweise den Kosinus.
Wollen wir sie mit anderen Funktionen vergleichen, so wäre es besser wenn sie in der Polynomform vorlägen, da mit dieser in der Algebra leicht umgegangen werden kann.
Hierzu benötigen wir das Taylorpolynom.
Dieses Polynom ist perfekt dazu geeignet komplizierte Funktionen mit hoher Genauigkeit anzunähern.
Das Polynom lässt sich nur dann anwenden, wenn die anzunähernde Funktion sich zureichend oft ableiten lässt, am besten unbeschränkt oft.
Für sowohl den Sinus als auch den Kosinus gilt dass, bei wiederholter Ableitung laufen sie durch die Werte $\sin(x)$, $\cos(x)$, $-\sin(x)$ und $-\cos(x)$ unendlich oft durch.
Das Taylorpolynom kann mit einer angegebenen Entwicklungsstelle $x_0$ den Wert jeder Funktion $f(x)$ an der Position $x$ durch die Formel
\begin{equation}
	T_n(x)=\frac{f(x_0)}{0!}\times(x-x_0)^0 + \frac{f^{(1)}(x_0)}{1!}\times(x-x_0)^1 +% \frac{f^{(2)}(x_0)}{2!}\times(x-x_0)^2 +
	\dots +\frac{f^{(n)}(x_0)}{n!}\times(x-x_0)^n
\end{equation} %Video von Mathepeter
wobei bei steigender Anzahl $n$ der Polynomglieder die Genauigkeit der Annäherung steigt.

Versuchen wir nun den $cos(x)$durch ein unendlich langes Taylorpolynom, eine sogenannte Taylorreihe, so sehr anzunähern, dass das Polynom getrost der Funktion gleichgesetzt werden kann.
Wir bestimmen $x_0=0$, da an dieser Stelle die Sinus- und Kosinuswerte einfach bestimmt werden können.
\[cos(x)=\frac{cos(0)}{0!}\times(x-0) + \frac{-sin(0)}{1!}\times(x-0) + \frac{-cos(0)}{2!}\times(x-0)^2+\frac{sin(0)}{3!}\times(x-0)^3+\dots\]
Dabei fällt auf, dass bei jedem zweiten Polynomglied im Zähler des Bruches ein positives oder negatives $sin(0)$ vorliegt.
Da $sin(0)=0$ ist, lässt sich das Polynom vereinfachen zu
\[cos(x)=1-\frac{x^2}{2!}+\frac{x^4}{4!}-\frac{x^6}{6!}+\frac{x^8}{8!}+\dots\]
Die Polynomformen anderer Funktionen erhält man nach analogem Verfahren, wie hier die Sinusfunktion
\[sin(x)=x-\frac{x}{1!}+\frac{x^3}{3!}-\frac{x^5}{5!}+\frac{x^7}{7!}+\dots\]
oder hier die $e$-Funktion
\[e^x=1+\frac{x}{1!}+\frac{x^2}{2!}+\frac{x^3}{3!}+\frac{x^4}{4!}+\dots\]

Mithilfe des Sinus und Kosinus können wir nun komplexe Zahlen als Polynome aufschreiben. Sehen wir uns der Einfachheit halber zuerst eine Zahl an, die auf dem Einheitskreis liegt und damit als $l=1$ hat.
\begin{equation}
	z=%\times(\cos(\varphi)+i\times\sin(\varphi))=
	\left(1-\frac{x^2}{2!}+\frac{x^4}{4!}-\frac{x^6}{6!}+\dots\right)+i\times\left(x-\frac{x}{1!}+\frac{x^3}{3!}-\frac{x^5}{5!}+\dots\right)
\end{equation}
Vergleicht man nun $z$ mit der Polynomform von $e^\varphi$ stellt man fest, dass sie gleich sind. 
In dem Taylorpolynom haben die Polynomglieder im Zähler ein $i^n\varphi^n$. Wir müssen nun bedenken, dass $i^2=-1$ ist und durch Umformungen und Umgruppierungen entsteht
\begin{multline}
	e^i\varphi=1+\frac{i\varphi}{1!}+\frac{i^2\varphi^2}{2!}+\frac{i^3\varphi^3}{3!}+\frac{i^4\varphi^4}{4!}+\dots=
	1+\frac{i\varphi}{1!}-\frac{\varphi^2}{2!}-\frac{i\varphi^3}{3!}+\frac{\varphi^4}{4!}+\dots=\\
	\left(1-\frac{\varphi^2}{2!}+\frac{\varphi^4}{4!}-\frac{\varphi^6}{6!}+\dots\right)+i\times\left(\varphi-\frac{\varphi}{1!}+\frac{x\varphi^3}{3!}-\frac{\varphi^5}{5!}+\dots\right) =\\
	\cos(\varphi) + i\times\sin(\varphi)=z
\end{multline}
Hierbei handelt es sich um die berühmte "{Eulersche Formel}", welche Besagt, dass
\[e^{i\varphi}=\cos(\varphi)+i\times\sin(\varphi)\]
Daraus folgt, dass komplexe Zahlen auf dem Einheitskreis durch den Term $e^{i\varphi}$ dargestellt werden können. Für andere komplexe Zahlen muss noch der Abstand $l$ vom Ursprung bedacht werden und für sie gilt
\[z=l\times e^{i\varphi}\]
Sind komplexe Zahlen in dieser Form geschrieben, so ist offensichtlich, dass Multiplikation von komplexen Zahlen durch eine Multiplikation der Längen und Addition der Winkel erreicht wird
\[z_1\times z_2=(l_1 \times e^{i\varphi_1}) \times (l_2 \times e^{i\varphi_2})=l_1 l_2 \times e^{i(\varphi_1+\varphi_2)}\]
Im Fall von Potenzieren von komplexen Zahlen gilt nach den Regeln der Multiplikation dann
\[l^n \times e^{i(n\varphi)}\]

Einen sehr anschaulichen Effekt hat diese Formel nun wenn wir versuchen ihre geometrische Bedeutung zu Interpretieren. Sie kann nämlich als die Formel für die Lösungen einer Gleichung der Form $z^n=h$ gesehen werden. Setzen wir als $h=1$ ein, so versuchen wir somit  die $n$-te Wurzel aus $1$ zu bestimmen. Benutzen wir nun Formel $l^n \times e^{i(n\varphi)}=1$, muss für jedes $z$ das $l=1$ sein, da $l^n=1$ gilt und $l$ nur positive Werte annehmen kann. Jede mögliche Lösung $z$ für diese Gleichung hat als $\varphi$ einen Winkel, der, wenn er mit $n$ multipliziert wird, ein Vielfaches von $2\pi$ ergibt. die Zahl $1$ hat in ihrer Darstellung als komplexe Zahl den Winkel $\alpha=0$ und somit muss $\varphi\times n$ auch nach beliebig vielen Umdrehungen wieder an der Realachse ankommen. Das $\varphi_1$ der ersten Lösung der Gleichung sowie $2\times\varphi_1$, $3\times\varphi_1$, $4\times\varphi_1$, $\dots$ sind alles Winkel bei denen dies gilt. Es gibt insgesamt $n$ verschiedene solche Winkel, eingeschlossen $\varphi=0$, da beginnend mit dem $n+1$sten der Winkel einem schon vorhandenen anderen $\varphi$ mit zusätzlichen
vollen Umdrehungen entspricht.
Dadurch, dass die Winkel gleich $0, \varphi, 2\times \varphi, 3 \times \varphi, \dots$ sind, verteilen sie sich in gleichem Winkelabstand voneinander auf dem Einheitskreis und bilden die Ecken eines gleichmäßiges $n$-Ecks.













\section{Beispiel der Anwendung}

Nun soll die Nützlichkeit der komplexen Zahlen an einer Aufgabe demonstriert werden, welche auf den Ersten Blick sich gar nicht mit Algebra lösen lässt, da sie aus der Geometrie stammt. Dieses Problem ist sehr einfach formuliert, und könnte noch aus der Antike stammen, als große Mathematik mit Stöcken in Sand gezeichnet wurde. Die Aufgabe Lautet:\\

\noindent \textit{In einen Einheitskreis wird ein regelmäßiges $n$-Eck eingezeichnet, sodass alle Ecken auf der Kreislinie liegen.
Eine der Ecken wird mit jeweils jeder anderen Ecke durch Strecken verbunden.
Was ist das Produkt der Längen dieser Strecken?
}\\

Aus dem Abschnitt (3...) kennen  wir Kreisteilungsgleichungen. Diese benötigt man zum Lösen der Aufgabe. Zuallererst jedoch muss die komplexe Zahlenebene festgelegt werden. 

Setzen wir den Ursprung unsres Koordinatensystems in den Mittelpunkt des Kreises und richten es so aus, dass die Achse der Realanteile durch eine der Ecken durchläuft und diese Ecke die Koordinaten $(1,0)$ hat.
Jene Ecke nennen wir $w_0$ und von dieser Ecke aus werden wir auch später die Linien zu den anderen Ecken ziehen.\\

Wenn wir nun alle Ecken als komplexe Zahlen ansehen, so können wir die Kreisteilungsgleichungen verwenden und kommen zum Schluss, dass
\begin{equation}
	z^n-1=0
\end{equation}

wobei $z$ eine $n$-te Wurzel von $1$ ist und somit auch ein Eckpunkt unseres $n$-Ecks.
Diese Ecken nennen wir nun $w_1$, $w_2$,$\dots$,$w_{n-1}$, dabei ist $w_n = w_0 = 1$.\\ 

Jeder Punkt in einem Koordinatensystem lässt sich auch als ein Vektor beginnend im Ursprung und endend in diesem Punkt darstellen. Komplexe Zahlen in ihrer geometrischer Darstellung werden genauso addieren und subtrahieren wie Vektoren und aus den Regeln der Subtraktion von Vektoren folgt, dass ein Vektor $\vec{v}$, welcher in $w_0$ beginnt und in $w_z$ endet, gleich $\vec{v}=w_z-w_0$ ist.
$w_0$ hat die Koordinaten (1,0).
Damit ist die Länge des Vektors, und somit der Strecke, welcher $w_0$ und $w_z$ verbindet gleich $|\vec{v}| = w_z - 1$.

In der Aufgabenstellung wird nach dem Produkt aller Verbindungsstrecken gefragt.
Damit ist dieses Produkt, welches wir nun $X$ nennen, gleich
\[X = |(w_1-1)|\times|(w_2-1)|\times\dots\times|(w_{n-1}-1)|\]
Jeder Faktor dieses Produkts ist der Betrag der jeweiligen Klammer, was bedeutet, dass man in jeder Klammer statt $(w_z-1)$ auch $(1-w_z)$ schreiben kann, ohne den Betrag zu verändern.
Das gesamte Produkt ist dann gleich 
\[X=|(1-w_1)|\times|(1-w_2)|\times\dots|(1-w_{n-1})|\]
Nach den Regeln der Multiplikation von Beträgen ist das Produkt der Beträge von Faktoren gleich dem Betrag des Proddukts dieser Faktoren. Also gilt
\begin{equation}\label{X}
	X=|(1-w_1)\times(1-w_2)\times\dots(1-w_{n-1})|
\end{equation}

An dieser Stelle lasst uns den Term, welcher sich innerhalb der Betragsstriche in $X$ liegt ansehen.
Nennen wir diesen Term $H$, so ist $X=|H|$ und
\begin{equation}\label{H}
	H=(1-w_1)\times(1-w_2)\times\dots(1-w_n-1)
\end{equation}

Erinnern wir uns, dass die Eckpunkte des Vielecks auch als Wurzel der Gleichung $z^n-1=0$ dargestellt werden können. Die erweiterte Fassung des Fundamentalsatzes der Algebra (\ref{eq:funda2}) besagt nun, dass
\[z^n-1=(z-w_0)\times(z-w_1)\times(z-w_2)\times\dots(z-w_{n-1})\]
da $w_0=1$ ist, folgt
\[z^n-1=(z-1)\times(z-w_1)\times(z-w_2)\times\dots(z-w_{n-1})\]
Diese Gleichung ist nahezu identisch mit $H$ bei $z=1$.
Der einzige Unterschied ist der Faktor $z-1$, welcher bei $H$ fehlt.
Möchten wir nun unsere Gleichung ohne diesen Faktor berechnen, so müssen wir sie durch $(z-1)$ teilen und es entsteht
\begin{equation}\label{frac}
	H=(z-w_1)\times(z-w_2)\times\dots(z-w_{n-1})=\frac{z^n-1}{z-1}
\end{equation}
bei $z=1$. Dies kann auch aufgeschrieben werden als
\[\frac{z^n-1}{z-1}=\frac{1-z^n}{1-z\hspace{2mm}}\]
Da wir versuchen, diese Gleichung bei $z=1$ zu lösen, müssen wir über einen Umweg gehen, ansonsten würde bei einem direkten Einsetzen der Nenner gleich $0$ sein.
Dabei fällt auf, dass $\frac{1-z^n}{1-z\hspace{1mm}}$ die Zusammenfassung einer geometrischen Reihe ist.
(Eine Geometrische Reihe $G_n$ ist ein Term der Form $a+aq+aq^2+\dots+aq^n$.)
Dass dies wahr ist lässt sich mithilfe der vollständigen Induktion zeigen.

"Wir werden beweisen, daß [sic!] für jeden Wert von $n$ [$\dots$]
\begin{equation}
	G_n=a+aq+aq^2+\dots+aq^n=a\frac{1-q^{n+1}}{1-q\hspace{6mm}}.
\end{equation} 
(Wir setzen $q\ne1$, lies: $q$ ungleich $1$,v voraus, da sonst die rechte Seite von [der obenstehenden Behauptung] keinen Sinn hätte.)

Diese Behauptung ist für $n=1$ sicher gültig; denn dann besagt sie, daß [sic!]
\[G_1=a+aq=\frac{a(1-q^2)}{1-q}=\frac{a(1+q)(1-q)}{1-q}=a(1+q).\]
Wenn wir nun annehmen, daß [sic!]
\[G_r=a+aq+\dots+aq^r=a\frac{1-q^{r+1}}{1-q},\]
dann finden wir als Folgerung daraus
\begin{multline}
	G-{r+1}=(a+aq+\dots+aq^r)+aq^{r+1}=G_r+aq^{r+1}=a\frac{1-q^{r+1}}{1-q\hspace{6mm}}+aq^{r+1}\\
	=a\frac{(1-q^{r+1})+q^{r+1}(1-q)}{1-q}=a\frac{1-q^{r+1}+q^{r+1}-q^{r+2}}{1-q}=a\frac{1-q^{r+2}}{1-q\hspace{6mm}}
\end{multline}
Dies ist aber gerade die Behauptung [, die wir beweisen,] für den Fall $n=r+1$. Damit ist der Beweis vollständig."(Zitat-Zeug)

Nun können wir $\frac{1-z^n}{1-z}$ als eine Geometrische Reihe aufschreiben.
\[1\times\frac{1-z^n}{1-z\hspace{2mm}}=1+z+z^2+\dots+z^{n-1}\]
Da $\frac{z^n-1}{z-1}=\frac{1-z^n}{1-z\hspace{1mm}}$ ist, gilt bei $z=1$
\[\frac{z^n-1}{z-1}=1+1+1^2+\dots+1^{n-1}=n\]
da die $1$ in dieser geometrischen Reihe insgesamt $n$ mal vorkommt.

Den Term $\frac{z^n-1}{z-1}$ haben wir ursprünglich aufgestellt, da er nach der Gleichung (\ref{H}) äquivalent zu $H$ ist. Damit ist
\[H=n\]
$H$ selber war der Term innerhalb der Betragsstriche von $X$, denn $X=|H|$.
Da nun $H=n$ ist mit $n$ als eine zwingend positive Zahl, lässt sich $H$ mit $X$ direkt gleichstellen und man schließt heraus, dass:
\[X=n\]
$X$ ist das Produkt aller Längen der Strecken, die eine Ecke des ursprünglichen Vielecks mit den verbinden. Somit ist die Aufgabe gelöst und jenes Produkt ist gleich der Anzahl der Ecken im Vieleck.





\section{Literaturverzeichnis}





\end{document} % конец документа

